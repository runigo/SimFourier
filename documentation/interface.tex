%
\chapter{Prise en main}
%
%%%%%%%%%%%%%%%%%%%%%%%%%
\section{Fonction initiale}
%%%%%%%%%%%%%%%%%%%%%%%%%
\newpage
\subsection{Boutons rotatifs}
%%%%%%%%%%%%%%%%%%%%%%%%%
\begin{tikzpicture}[
 % scale=\scl,
  %panneauControles/.style={black!10!brown!10!,draw=black!60!brown!40!,thick},
  panneauControles/.style={black!10!brown!10!},
  panneauOnglet/.style={bottom color=black!60!brown!60!, top color=black!20!brown!20!}]

\begin{scope}[xshift=-7 cm,yshift=0.0cm]

                   %  Onglets et panneau
  \begin{scope}[xshift=0 cm,yshift=20cm]
    \fill[gray,draw=gray!10!] (0.1, 0.6) rectangle (2.2,-19.1);
    \fill[panneauControles]
      (0.2, 0.5) -- (2.2, 0.5) -- (2.2, -19) -- (0.2, -19) -- cycle;
  \end{scope}
    \draw (1.8,20) node [black]{ENVELOPPE};

%%%%%%%%%%%%%%%%%%%%%%%%%%%%%%  FRÉQUENCE  ENVELOPPE

  \begin{scope}[xshift=1.9 cm,yshift=17.8cm, scale=0.5]
    

%%%%%%%%%%%%%%%%%%%%%%%%%%%%%%    BOUTONS ROTATIF
%      Carrés gris
  \fill[gray!50!] (0,0) rectangle (3,3);
  \fill[gray] (1,0) rectangle (3,2);
  \fill[gray!50!black] (2,0) rectangle (3,1);
%%%%%%%%%


\newcommand{\clock}[4]{%
  \begin{scope}[xshift=2.25*#1cm,yshift=2.25*#2cm]
  % \filldraw [fill=#3, line width=1.6pt] (0,0) circle (1cm); % simple fill (unused)
  % \draw[fill] (0,0) circle (1mm); % border (unused)
  \shadedraw [inner color=#3!30!white, outer color=#3!90!black, 
    line width=1.6pt] (0,0) circle (1cm); % disk with shadow and border
  \foreach \angle in {0, 30, ..., 330} 
    \draw[line width=1pt] (\angle:0.82cm) -- (\angle:1cm);
  \foreach \angle in {0,90,180,270}
    \draw[line width=1.3pt] (\angle:0.75cm) -- (\angle:1cm);
  \draw[line width=1.6pt] (0,0) -- (90-30*#4:0.6cm); % the hand 
  \end{scope}
}
% Default on latex.ltx: \fboxsep = 3pt \fboxrule = .4pt
\fboxsep = 3pt \fboxrule = 1.5pt 
\newcounter{itimesj}
\fbox{%
\begin{tikzpicture}[scale=0.5]
\begin{scope}[line cap=round]
\foreach \i in {0,...,11}
  \foreach \j in {0,...,11} {
    % We use round by caution, because we are getting real numbers
    \pgfmathparse{round(mod(\i*\j,12))} 
    % \pgfmathresult = 3.0 gives itimesj = 3
    \pgfmathsetcounter{itimesj}{\pgfmathresult} 
    \clock{\i}{-\j}{clock\theitimesj}{\i*\j}; 
  }
\end{scope}

    \draw (-2.7,2.6) node [black]{P$_0$};
  \end{scope}
  \begin{scope}[xshift=1.9 cm,yshift=16cm, scale=0.5]
    

%%%%%%%%%%%%%%%%%%%%%%%%%%%%%%    BOUTONS ROTATIF
%      Carrés gris
  \fill[gray!50!] (0,0) rectangle (3,3);
  \fill[gray] (1,0) rectangle (3,2);
  \fill[gray!50!black] (2,0) rectangle (3,1);
%%%%%%%%%


\newcommand{\clock}[4]{%
  \begin{scope}[xshift=2.25*#1cm,yshift=2.25*#2cm]
  % \filldraw [fill=#3, line width=1.6pt] (0,0) circle (1cm); % simple fill (unused)
  % \draw[fill] (0,0) circle (1mm); % border (unused)
  \shadedraw [inner color=#3!30!white, outer color=#3!90!black, 
    line width=1.6pt] (0,0) circle (1cm); % disk with shadow and border
  \foreach \angle in {0, 30, ..., 330} 
    \draw[line width=1pt] (\angle:0.82cm) -- (\angle:1cm);
  \foreach \angle in {0,90,180,270}
    \draw[line width=1.3pt] (\angle:0.75cm) -- (\angle:1cm);
  \draw[line width=1.6pt] (0,0) -- (90-30*#4:0.6cm); % the hand 
  \end{scope}
}
% Default on latex.ltx: \fboxsep = 3pt \fboxrule = .4pt
\fboxsep = 3pt \fboxrule = 1.5pt 
\newcounter{itimesj}
\fbox{%
\begin{tikzpicture}[scale=0.5]
\begin{scope}[line cap=round]
\foreach \i in {0,...,11}
  \foreach \j in {0,...,11} {
    % We use round by caution, because we are getting real numbers
    \pgfmathparse{round(mod(\i*\j,12))} 
    % \pgfmathresult = 3.0 gives itimesj = 3
    \pgfmathsetcounter{itimesj}{\pgfmathresult} 
    \clock{\i}{-\j}{clock\theitimesj}{\i*\j}; 
  }
\end{scope}

    \draw (-2.7,2.6) node [black]{P$_1$};
  \end{scope}
  
%%%%%%%%%%%%%%%%%%%%%%%%%%%%%%  MOYENNE  ET  SYMÉTRIE


  \begin{scope}[xshift=1.9 cm,yshift=13.5cm, scale=0.5]
    

%%%%%%%%%%%%%%%%%%%%%%%%%%%%%%    BOUTONS ROTATIF
%      Carrés gris
  \fill[gray!50!] (0,0) rectangle (3,3);
  \fill[gray] (1,0) rectangle (3,2);
  \fill[gray!50!black] (2,0) rectangle (3,1);
%%%%%%%%%


\newcommand{\clock}[4]{%
  \begin{scope}[xshift=2.25*#1cm,yshift=2.25*#2cm]
  % \filldraw [fill=#3, line width=1.6pt] (0,0) circle (1cm); % simple fill (unused)
  % \draw[fill] (0,0) circle (1mm); % border (unused)
  \shadedraw [inner color=#3!30!white, outer color=#3!90!black, 
    line width=1.6pt] (0,0) circle (1cm); % disk with shadow and border
  \foreach \angle in {0, 30, ..., 330} 
    \draw[line width=1pt] (\angle:0.82cm) -- (\angle:1cm);
  \foreach \angle in {0,90,180,270}
    \draw[line width=1.3pt] (\angle:0.75cm) -- (\angle:1cm);
  \draw[line width=1.6pt] (0,0) -- (90-30*#4:0.6cm); % the hand 
  \end{scope}
}
% Default on latex.ltx: \fboxsep = 3pt \fboxrule = .4pt
\fboxsep = 3pt \fboxrule = 1.5pt 
\newcounter{itimesj}
\fbox{%
\begin{tikzpicture}[scale=0.5]
\begin{scope}[line cap=round]
\foreach \i in {0,...,11}
  \foreach \j in {0,...,11} {
    % We use round by caution, because we are getting real numbers
    \pgfmathparse{round(mod(\i*\j,12))} 
    % \pgfmathresult = 3.0 gives itimesj = 3
    \pgfmathsetcounter{itimesj}{\pgfmathresult} 
    \clock{\i}{-\j}{clock\theitimesj}{\i*\j}; 
  }
\end{scope}

    \draw (-2.7,2.6) node [black]{S};
  \end{scope}
  \begin{scope}[xshift=1.9 cm,yshift=11.7cm, scale=0.5]
    

%%%%%%%%%%%%%%%%%%%%%%%%%%%%%%    BOUTONS ROTATIF
%      Carrés gris
  \fill[gray!50!] (0,0) rectangle (3,3);
  \fill[gray] (1,0) rectangle (3,2);
  \fill[gray!50!black] (2,0) rectangle (3,1);
%%%%%%%%%


\newcommand{\clock}[4]{%
  \begin{scope}[xshift=2.25*#1cm,yshift=2.25*#2cm]
  % \filldraw [fill=#3, line width=1.6pt] (0,0) circle (1cm); % simple fill (unused)
  % \draw[fill] (0,0) circle (1mm); % border (unused)
  \shadedraw [inner color=#3!30!white, outer color=#3!90!black, 
    line width=1.6pt] (0,0) circle (1cm); % disk with shadow and border
  \foreach \angle in {0, 30, ..., 330} 
    \draw[line width=1pt] (\angle:0.82cm) -- (\angle:1cm);
  \foreach \angle in {0,90,180,270}
    \draw[line width=1.3pt] (\angle:0.75cm) -- (\angle:1cm);
  \draw[line width=1.6pt] (0,0) -- (90-30*#4:0.6cm); % the hand 
  \end{scope}
}
% Default on latex.ltx: \fboxsep = 3pt \fboxrule = .4pt
\fboxsep = 3pt \fboxrule = 1.5pt 
\newcounter{itimesj}
\fbox{%
\begin{tikzpicture}[scale=0.5]
\begin{scope}[line cap=round]
\foreach \i in {0,...,11}
  \foreach \j in {0,...,11} {
    % We use round by caution, because we are getting real numbers
    \pgfmathparse{round(mod(\i*\j,12))} 
    % \pgfmathresult = 3.0 gives itimesj = 3
    \pgfmathsetcounter{itimesj}{\pgfmathresult} 
    \clock{\i}{-\j}{clock\theitimesj}{\i*\j}; 
  }
\end{scope}

    \draw (-2.7,2.6) node [black]{\Large{$\phi$}};
  \end{scope}

%%%%%%%%%%%%%%%%%%%%%%%%%%%%%%  DÉCALAGE ET AMPLITUDE

  \begin{scope}[xshift=1.9 cm,yshift=9cm, scale=0.5]
    

%%%%%%%%%%%%%%%%%%%%%%%%%%%%%%    BOUTONS ROTATIF
%      Carrés gris
  \fill[gray!50!] (0,0) rectangle (3,3);
  \fill[gray] (1,0) rectangle (3,2);
  \fill[gray!50!black] (2,0) rectangle (3,1);
%%%%%%%%%


\newcommand{\clock}[4]{%
  \begin{scope}[xshift=2.25*#1cm,yshift=2.25*#2cm]
  % \filldraw [fill=#3, line width=1.6pt] (0,0) circle (1cm); % simple fill (unused)
  % \draw[fill] (0,0) circle (1mm); % border (unused)
  \shadedraw [inner color=#3!30!white, outer color=#3!90!black, 
    line width=1.6pt] (0,0) circle (1cm); % disk with shadow and border
  \foreach \angle in {0, 30, ..., 330} 
    \draw[line width=1pt] (\angle:0.82cm) -- (\angle:1cm);
  \foreach \angle in {0,90,180,270}
    \draw[line width=1.3pt] (\angle:0.75cm) -- (\angle:1cm);
  \draw[line width=1.6pt] (0,0) -- (90-30*#4:0.6cm); % the hand 
  \end{scope}
}
% Default on latex.ltx: \fboxsep = 3pt \fboxrule = .4pt
\fboxsep = 3pt \fboxrule = 1.5pt 
\newcounter{itimesj}
\fbox{%
\begin{tikzpicture}[scale=0.5]
\begin{scope}[line cap=round]
\foreach \i in {0,...,11}
  \foreach \j in {0,...,11} {
    % We use round by caution, because we are getting real numbers
    \pgfmathparse{round(mod(\i*\j,12))} 
    % \pgfmathresult = 3.0 gives itimesj = 3
    \pgfmathsetcounter{itimesj}{\pgfmathresult} 
    \clock{\i}{-\j}{clock\theitimesj}{\i*\j}; 
  }
\end{scope}

    \draw (-2.7,2.6) node [black]{A};
  \end{scope}
  \begin{scope}[xshift=1.9 cm,yshift=7.2cm, scale=0.5]
    

%%%%%%%%%%%%%%%%%%%%%%%%%%%%%%    BOUTONS ROTATIF
%      Carrés gris
  \fill[gray!50!] (0,0) rectangle (3,3);
  \fill[gray] (1,0) rectangle (3,2);
  \fill[gray!50!black] (2,0) rectangle (3,1);
%%%%%%%%%


\newcommand{\clock}[4]{%
  \begin{scope}[xshift=2.25*#1cm,yshift=2.25*#2cm]
  % \filldraw [fill=#3, line width=1.6pt] (0,0) circle (1cm); % simple fill (unused)
  % \draw[fill] (0,0) circle (1mm); % border (unused)
  \shadedraw [inner color=#3!30!white, outer color=#3!90!black, 
    line width=1.6pt] (0,0) circle (1cm); % disk with shadow and border
  \foreach \angle in {0, 30, ..., 330} 
    \draw[line width=1pt] (\angle:0.82cm) -- (\angle:1cm);
  \foreach \angle in {0,90,180,270}
    \draw[line width=1.3pt] (\angle:0.75cm) -- (\angle:1cm);
  \draw[line width=1.6pt] (0,0) -- (90-30*#4:0.6cm); % the hand 
  \end{scope}
}
% Default on latex.ltx: \fboxsep = 3pt \fboxrule = .4pt
\fboxsep = 3pt \fboxrule = 1.5pt 
\newcounter{itimesj}
\fbox{%
\begin{tikzpicture}[scale=0.5]
\begin{scope}[line cap=round]
\foreach \i in {0,...,11}
  \foreach \j in {0,...,11} {
    % We use round by caution, because we are getting real numbers
    \pgfmathparse{round(mod(\i*\j,12))} 
    % \pgfmathresult = 3.0 gives itimesj = 3
    \pgfmathsetcounter{itimesj}{\pgfmathresult} 
    \clock{\i}{-\j}{clock\theitimesj}{\i*\j}; 
  }
\end{scope}

    \draw (-2.7,2.6) node [black]{\Large{$\Delta$}};
  \end{scope}

%%%%%%%%%%%%%%%%%%%%%%%%%%%%%%  PORTEUSE
    \draw (1.7,6) node [black]{PORTEUSE};
%%%%%%%%%%%%%%%%%%%%%%%%%%%%%%  FRÉQUENCE  PORTEUSE
  \begin{scope}[xshift=1.9 cm,yshift=3.8cm, scale=0.5]
    

%%%%%%%%%%%%%%%%%%%%%%%%%%%%%%    BOUTONS ROTATIF
%      Carrés gris
  \fill[gray!50!] (0,0) rectangle (3,3);
  \fill[gray] (1,0) rectangle (3,2);
  \fill[gray!50!black] (2,0) rectangle (3,1);
%%%%%%%%%


\newcommand{\clock}[4]{%
  \begin{scope}[xshift=2.25*#1cm,yshift=2.25*#2cm]
  % \filldraw [fill=#3, line width=1.6pt] (0,0) circle (1cm); % simple fill (unused)
  % \draw[fill] (0,0) circle (1mm); % border (unused)
  \shadedraw [inner color=#3!30!white, outer color=#3!90!black, 
    line width=1.6pt] (0,0) circle (1cm); % disk with shadow and border
  \foreach \angle in {0, 30, ..., 330} 
    \draw[line width=1pt] (\angle:0.82cm) -- (\angle:1cm);
  \foreach \angle in {0,90,180,270}
    \draw[line width=1.3pt] (\angle:0.75cm) -- (\angle:1cm);
  \draw[line width=1.6pt] (0,0) -- (90-30*#4:0.6cm); % the hand 
  \end{scope}
}
% Default on latex.ltx: \fboxsep = 3pt \fboxrule = .4pt
\fboxsep = 3pt \fboxrule = 1.5pt 
\newcounter{itimesj}
\fbox{%
\begin{tikzpicture}[scale=0.5]
\begin{scope}[line cap=round]
\foreach \i in {0,...,11}
  \foreach \j in {0,...,11} {
    % We use round by caution, because we are getting real numbers
    \pgfmathparse{round(mod(\i*\j,12))} 
    % \pgfmathresult = 3.0 gives itimesj = 3
    \pgfmathsetcounter{itimesj}{\pgfmathresult} 
    \clock{\i}{-\j}{clock\theitimesj}{\i*\j}; 
  }
\end{scope}

    \draw (-2.7,2.6) node [black]{P$_0$};
  \end{scope}
  \begin{scope}[xshift=1.9 cm,yshift=2cm, scale=0.5]
    

%%%%%%%%%%%%%%%%%%%%%%%%%%%%%%    BOUTONS ROTATIF
%      Carrés gris
  \fill[gray!50!] (0,0) rectangle (3,3);
  \fill[gray] (1,0) rectangle (3,2);
  \fill[gray!50!black] (2,0) rectangle (3,1);
%%%%%%%%%


\newcommand{\clock}[4]{%
  \begin{scope}[xshift=2.25*#1cm,yshift=2.25*#2cm]
  % \filldraw [fill=#3, line width=1.6pt] (0,0) circle (1cm); % simple fill (unused)
  % \draw[fill] (0,0) circle (1mm); % border (unused)
  \shadedraw [inner color=#3!30!white, outer color=#3!90!black, 
    line width=1.6pt] (0,0) circle (1cm); % disk with shadow and border
  \foreach \angle in {0, 30, ..., 330} 
    \draw[line width=1pt] (\angle:0.82cm) -- (\angle:1cm);
  \foreach \angle in {0,90,180,270}
    \draw[line width=1.3pt] (\angle:0.75cm) -- (\angle:1cm);
  \draw[line width=1.6pt] (0,0) -- (90-30*#4:0.6cm); % the hand 
  \end{scope}
}
% Default on latex.ltx: \fboxsep = 3pt \fboxrule = .4pt
\fboxsep = 3pt \fboxrule = 1.5pt 
\newcounter{itimesj}
\fbox{%
\begin{tikzpicture}[scale=0.5]
\begin{scope}[line cap=round]
\foreach \i in {0,...,11}
  \foreach \j in {0,...,11} {
    % We use round by caution, because we are getting real numbers
    \pgfmathparse{round(mod(\i*\j,12))} 
    % \pgfmathresult = 3.0 gives itimesj = 3
    \pgfmathsetcounter{itimesj}{\pgfmathresult} 
    \clock{\i}{-\j}{clock\theitimesj}{\i*\j}; 
  }
\end{scope}

    \draw (-2.7,2.6) node [black]{P$_1$};
  \end{scope}
\end{scope}
\end{tikzpicture}

%%%%%%%%%%%%%%%%%%%%%%%%%
\section{Clavier}
%%%%%%%%%%%%%%%%%%%%%%%%%\mt{}
Le clavier permet de modifier les paramètres physiques. La fenêtre graphique doit être active, le terminal affiche les informations.
%
\footnotesize
\begin{center}
\begin{tabular}{cccccccccc}
%\sffamily
%\rmfamily
\sf A &\sf Z &\sf E &\sf R &\sf T &\sf Y &\sf U &\sf I &\sf O &\sf P \\
 & env 0 & env 1 & env -1 &  &  & co -1 & co 1 & co 0 & Fréquence \\
\sf Q &\sf S &\sf D &\sf F &\sf G &\sf H &\sf J &\sf K &\sf L &\sf M \\
 &  &  &  &  &  &  &  &  &  \\
\sf W &\sf X &\sf C &\sf V &\sf B &\sf N &  &  &  & \\
 &  &  &  &  &  &  &  &  & \\
\end{tabular}
\end{center}
\vspace{.3cm}
%
\normalsize
\begin{itemize}[leftmargin=2cm, label=\ding{32}, itemsep=0pt]%\end{itemize}
\item Caractéristiques de l'enveloppe

\begin{tabular}{ccccc}
%\sffamily
%\rmfamily
\sf A &\sf Z &\sf E &\sf R &\sf T  \\
env 0 & env 1 & env -1 &  &   \\
\sf Q &\sf S &\sf D &\sf F &\sf G  \\
 &  &  &  &   \\
 & & & \sf V   & \\
 & & &  &  \\
\end{tabular}
%
\end{itemize}
\begin{itemize}[leftmargin=2cm, label=\ding{32}, itemsep=0pt]%\end{itemize}
\item Paramètres graphique

\hspace{3cm}
\begin{tabular}{cccccccccc}
%\sffamily
%\rmfamily
 & & & &\sf Y \\
 & & & & Phi \\
 & & & &\sf H \\
 & & & &   & & &  & \\
\sf W &\sf X &\sf C &\sf V &\sf B &\sf N  \\
 &  &  &  & ech - & ech + \\
\end{tabular}
%
\end{itemize}
\begin{itemize}[leftmargin=2cm, label=\ding{32}, itemsep=0pt]%\end{itemize}
\item Caractéristique de la porteuse

\hspace{9cm}
\begin{tabular}{ccccc}
%\sffamily
%\rmfamily
\sf U &\sf I &\sf O &\sf P \\
 co -1 & co 1 & co 0 & Fréquence \\
\sf J &\sf K &\sf L &\sf M \\
  &  & carré & moinsF \\
\end{tabular}
%
\end{itemize}
%\vspace{.3cm}
Les touches de fonctions donnent un certain nombre de contrôles et d'information:
%
\begin{center}
\begin{tabular}{ccccc ccccc cc}
\multicolumn{4}{|c|}{Contrôles} & \multicolumn{4}{c}{Information} & \multicolumn{4}{|c|}{Contrôles}\\
\sf F1 &\sf F2 &\sf F3 &\sf F4 &\sf F5 &\sf F6 &\sf F7 &\sf F8 &\sf F9 &\sf F10 &\sf F11 &\sf F12 \\
\multicolumn{4}{|c|}{Équation simulé (SiCF)} & \multicolumn{4}{c}{Énergie, graphe} & \multicolumn{4}{|c|}{Vitesse de la simulation}\\
\end{tabular}
\end{center}
%
Le choix de l'équation simulée est spécifique à SiCF. {\sf F5} dresse un bilan énergétique. {\sf F6} affiche les paramètres physiques du système
\begin{center}
\begin{tabular}{cccccc}
\sf F1 &\sf F2 &\sf F3 &\sf F4 &\sf F5 &\sf F6\\
Pendules & Harmoniques & Corde & asymétrique & Énergie & Système \\
\end{tabular}
\end{center}
%
{\sf F8} permet de modifier  le graphisme de SiCP. {\sf F9} et {\sf F12} modifient rapidement le rythme de la simulation, {\sf F10} et {\sf F11} la modifie modéremment. La touche {\sf Entrée} change le mode avec ou sans attente, en mode avec attente, l'appuie sur une touche permet l'évolution du système.
\begin{center}
\begin{tabular}{cccccc}
\sf F8 &\sf F9 &\sf F10 &\sf F11 &\sf F12 & \sf Entrée \\
Support (SiCP) & -Sim & -Sim & +Sim & +Sim & mode\\
\end{tabular}
\end{center}
%\section{Détails des contrôles}
%
%\item {\bf } : \sf{} : 
%%%%%%%%%%%%%%%%%%%%%%%%%%%%%%%%%%%%%%%%%%%%%%%%%%%%%%%%%
%
%%%%%%%%%%%%%%%%%%%%%%%%%%%%%%%%%%%%%%%%%%%%%%%%%%%%%%%%%%%%%%%%%%%%%%%%%%%%%%%%%%%%%%%%%%%%%
