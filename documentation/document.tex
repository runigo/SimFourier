\documentclass[12pt, a4paper]{report}
%\documentclass[11pt, a4paper]{article}

%====================== PACKAGES ======================
\usepackage[french]{babel}

\frenchbsetup{StandardLists=true}
\usepackage{enumitem}
\usepackage{pifont}

\usepackage[utf8x]{inputenc}
%\usepackage[latin1]{inputenc}

%pour gérer les positionnement d'images
\usepackage{float}
\usepackage{amsmath}
\usepackage{amssymb}
\DeclareMathOperator{\dt}{dt}
\usepackage{graphicx}
%\usepackage{tabularx}
\usepackage[colorinlistoftodos]{todonotes}
\usepackage{url}

%pour les informations sur un document compilé en PDF et les liens externes / internes
\usepackage[pdfborder=0]{hyperref}
\hypersetup{
	colorlinks = true
	}

%pour la mise en page des tableaux
\usepackage{array}
\usepackage{tabularx}
\usepackage{multirow}
\usepackage{multicol}
\setlength{\columnsep}{50pt}

%pour utiliser \floatbarrier
%\usepackage{placeins}
%\usepackage{floatrow}

%espacement entre les lignes
\usepackage{setspace}

%modifier la mise en page de l'abstract
\usepackage{abstract}

%police et mise en page (marges) du document
\usepackage[T1]{fontenc}
\usepackage[top=2cm, bottom=2cm, left=2cm, right=2cm]{geometry}

%Pour les galerie d'images
\usepackage{subfig}

\usepackage{pdfpages}

\usepackage{tikz}
\usetikzlibrary{trees}
\usetikzlibrary{decorations.pathmorphing}
\usetikzlibrary{decorations.markings}
\usetikzlibrary{decorations.pathreplacing,calligraphy}
%\usetikzlibrary{decorations}
\usetikzlibrary{angles, quotes}
\usepackage{verbatim}

\usepackage{appendix}

\usepackage{comment}

\usepackage{xcolor}

%\PreviewEnvironment{tikzpicture}
%\setlength\PreviewBorder{0pt}%

%====================== INFORMATION ET REGLES ======================

%rajouter les numérotation pour les \paragraphe et \subparagraphe
\setcounter{secnumdepth}{4}
\setcounter{tocdepth}{4}

\hypersetup{							% Information sur le document
pdfauthor = {Stephan Runigo},			% Auteurs
pdftitle = {SimFourier},			% Titre du document
pdfsubject = {Documentation},		% Sujet
pdfkeywords = {Transformée de Fourier, Fourier},	% Mots-clefs
pdfstartview={FitH}}	% ajuste la page à la largeur de l'écran
%pdfcreator = {MikTeX},% Logiciel qui a crée le document
%pdfproducer = {} % Société avec produit le logiciel

%======================== DEFINITION COMMANDES ========================
\newcommand{\mt}[1]{\text{#1}}
\newcommand{\ul}[1]{\underline{#1}}
\newcommand{\mc}[1]{\mathcal{#1}}
\newcommand{\pt}[1]{\dot{\text{#1}}}
%======================== DEBUT DU DOCUMENT ========================
%
\begin{document}
%
%régler l'espacement entre les lignes
\newcommand{\HRule}{\rule{\linewidth}{0.5mm}}
%
% Titre, résumé, ... %
%\input{./presentation/presentation.tex}
%
% Table des matières
\tableofcontents
\thispagestyle{empty}
\setcounter{page}{0}
%
%espacement entre les lignes des tableaux
\renewcommand{\arraystretch}{1.5}
%
%====================== INCLUSION DES CHAPITRES ======================
%
~
\thispagestyle{empty}
%recommencer la numérotation des pages à "1"
\setcounter{page}{0}
\newpage
%

%%%%%%%%%%%%%%%%%%%%%
\chapter{Numérisation}
%%%%%%%%%%%%%%%%%%%%%

%%%%%%%%%%%%%%%%%%%%%
\section{Discrétisation des équations différentielles} %%%%%%%%%  \mt{}
%%%%%%%%%%%%%%%%%%%%%
%
La discrétisation des l'équations différentielles se fait à l'aide de l'algorithme de Verlet. Cet algorithme consiste à symétriser la dérivée par rapport au temps puis d'obtenir une expression de $\psi(\mt{x},\mt{t}+\dt)$ en fonction de $\psi(\mt{x},\mt{t})$ et $\psi(\mt{x},\mt{t}-\dt)$. Cette expression permet de simuler de proche en proche le comportement du système physique. La solution discrète se rapproche de la solution analytique si la valeur de dt est convenablement choisie. En dehors d'un certain encadrement de dt, la solution discrète s'éloigne de la solution analytique.
%
\subsection{Discrétisation des dérivées}
%
\subsubsection{Dérivé symétrisée}
Par définition, la dérivé symétrisée est :
\[
\frac{\mt{d}\psi(\mt{t})}{\dt}=\frac{\psi(\mt{t}+\dt)-\psi(\mt{t}-\dt)}{2\dt}
\]
On en déduit l'expression suivante de la dérivée seconde :
\[
\frac{\mt{d}^2\psi(\mt{t})}{\dt^2}=\frac{\psi (\mt{t}+2\dt)-\psi (\mt{t})-\psi (\mt{t})+\psi(\mt{t}-2\dt)}{4\dt^2}
\]
Le changement de variable dt' = 2 dt simplifie cette expression :
\[
\frac{\mt{d}^2\psi(\mt{t})}{\dt^2}=\frac{\psi(\mt{t}+\dt)-2\psi(\mt{t})+\psi(\mt{t}-\dt)}{\dt^2}
\]

\subsection{Discrétisation de l'équation de Schrödinger}
%
On introduit les dérivées symétrisées dans l'équation de Schrödinger :
\[
\mt{i}\hbar\ \frac{\partial\psi}{\partial\mt{t}} =
\frac{-\hbar^2}{2\mt{m}}\ \frac{\partial^2\psi}{\partial\mt{x}^2}
+\mt{V(x)}\psi
\]
%
ce qui donne :
\[
\mt{i}\hbar\ \frac{\psi(\mt{x},\mt{t}+\dt)-\psi(\mt{x},\mt{t}-\dt)}{2\dt} =
\frac{-\hbar^2}{2\mt{m}}\ \frac{\psi(\mt{x}+\mt{dx},\mt{t})-2\psi(\mt{x},\mt{t})+\psi(\mt{x}-\mt{dx},\mt{t})}{\mt{dx}^2}
+\mt{V(x)}\psi(\mt{x},\mt{t})
\]
%
soit
\[
\psi(\mt{x},\mt{t}+\dt)-\psi(\mt{x},\mt{t}-\dt)=
\frac{\mt{i}\hbar\dt}{\mt{m}}\ \frac{\psi(\mt{x}+\mt{dx},\mt{t})-2\psi(\mt{x},\mt{t})+\psi(\mt{x}-\mt{dx},\mt{t})}{\mt{dx}^2}
-\frac{2\mt{i}\dt}{\hbar}\ \mt{V(x)}\psi(\mt{x},\mt{t})
\]
%ou encore
%\[
%\psi(\mt{x},\mt{t}+\dt)=\psi(\mt{x},\mt{t}-\dt)+
%\frac{\mt{i}\hbar\dt}{\mt{m }\mt{dx}^2}\ \Big(\psi(\mt{x}+\mt{dx},\mt{t})+\psi(\mt{x}-\mt{dx},\mt{t})\Big)
%-\Big(\frac{2\mt{i}\hbar\dt}{\mt{m }\mt{dx}^2}+\frac{2\mt{i}\dt}{\hbar}\ \mt{V(x)}\Big)\psi(\mt{x},\mt{t})
%\]
ou
\[
\psi(\mt{x},\mt{t}+\dt)=\psi(\mt{x},\mt{t}-\dt)+
\frac{\mt{i}\hbar\dt}{\mt{m }\mt{dx}^2}\ \Big[\psi(\mt{x}+\mt{dx},\mt{t})+\psi(\mt{x}-\mt{dx},\mt{t})
-\big(2+\frac{2\mt{m}}{\hbar^2}\ \mt{V(x)}\big)\psi(\mt{x},\mt{t})\Big]
\]

On pose alors : Dt = $\frac{\mt{i}\hbar\dt}{\mt{m }\mt{dx}^2}$ et v(x)$=2+\frac{2\mt{m}}{\hbar^2}\ \mt{V(x)}$, et on obtient :
\[
\psi(\mt{x},\mt{t}+\dt)=\psi(\mt{x},\mt{t}-\dt)+
\Big[\psi(\mt{x}+\mt{dx},\mt{t})+\psi(\mt{x}-\mt{dx},\mt{t})
-\mt{v(x) }\psi(\mt{x},\mt{t})\Big]\mt{Dt}
\]

%%%%%%%%%%%%%%%%%%%%%
\section{Énergie potentielle} %%%%%%%%%  \mt{}
%%%%%%%%%%%%%%%%%%%%%
%
Le mode Ep permet d'enregistrer une fonction énergie potentielle : le passage à ce mode enregistre la fonction initiale. La partie réelle de cette fonction est utilisée comme fonction énergie potentielle, la partie imaginaire est utilisé commme potentiel réduit tel que définie précédement.
%\subsection{}
%



%%%%%%%%%%%%%%%%%%%%%
\chapter{Transformation de fourier}
%%%%%%%%%%%%%%%%%%%%%

%%%%%%%%%%%%%%%%%%%%%%%%%
\section{Série de fourier}
%%%%%%%%%%%%%%%%%%%%%%%%%\mt{}
Une fonction périodique (de période $T$) est égale à une somme discrète de sinusoïde :
\[
f_T(x)=a_0 + \sum_{n=1}^\infty \left( a_n \cos \frac{2 \pi n x}{T} + b_n \sin \frac{2 \pi n x}{T} \right)
\]
$a_n$ et $b_n$ sont les coefficients de fourier de $f_T(x)$.
En posant $\omega=\frac{2\pi}{T}$,
les coefficients de fourier sont donnés par :
\[
a_0=\frac{1}{T}\int_0^Tf(t)\mt{ d}t \hspace{3cm} \mt{et pour n>0 :}
\]
\[
a_n=\frac{2}{T}\int_0^Tf(t)\cos n \omega t\mt{ d}t \hspace{2cm} \mt{et} \hspace{2cm} b_n=\frac{2}{T}\int_0^Tf(t)\sin n \omega t\mt{ d}t
\]
%%%%%%%%%%%%%%%%%%%%%%%%%
\section{Transformé de fourier}
%%%%%%%%%%%%%%%%%%%%%%%%%
Une fonction (respectant certaines conditions) est égale à une somme continue d'exponentiel complexe :
\[
f(x) = \int_{-\infty}^\infty e^{2 i \pi \nu x}\widehat{f}(\nu) d\nu
\]
$\widehat{f}(\nu)$ est la transformé de fourier de $f(x)$. Elle est donnée par :
\[
\widehat{f}(\nu) = \int_{-\infty}^\infty e^{-2 i \pi \nu x}f(x) dx
\]
%%%%%%%%%%%%%%%%%%%%%%%%%%%%%%%%%%%%%%%%%%%%%%%%%%%%%%%%%%%%%%%%%%%%%%%%%%%%%%%%%%%%%

%
\chapter{Commande du clavier SimFourier}
%
Le clavier permet de modifier les paramètres physiques. La fenêtre graphique doit être active, le terminal affiche les informations.
%
\footnotesize
\begin{center}
\begin{tabular}{cccccccccc}
%\sffamily
%\rmfamily
\sf A &\sf Z &\sf E &\sf R &\sf T &\sf Y &\sf U &\sf I &\sf O &\sf P \\
 & env 0 & env 1 & env -1 &  &  & co -1 & co 1 & co 0 & Fréquence \\
\sf Q &\sf S &\sf D &\sf F &\sf G &\sf H &\sf J &\sf K &\sf L &\sf M \\
 &  &  &  &  &  &  &  &  &  \\
\sf W &\sf X &\sf C &\sf V &\sf B &\sf N &  &  &  & \\
 &  &  &  &  &  &  &  &  & \\
\end{tabular}
\end{center}
\vspace{.3cm}
%
\normalsize
\begin{itemize}[leftmargin=2cm, label=\ding{32}, itemsep=0pt]%\end{itemize}
\item Caractéristiques de l'enveloppe

\begin{tabular}{ccccc}
%\sffamily
%\rmfamily
\sf A &\sf Z &\sf E &\sf R &\sf T  \\
env 0 & env 1 & env -1 &  &   \\
\sf Q &\sf S &\sf D &\sf F &\sf G  \\
 &  &  &  &   \\
 & & & \sf V   & \\
 & & &  &  \\
\end{tabular}
%
\end{itemize}
\begin{itemize}[leftmargin=2cm, label=\ding{32}, itemsep=0pt]%\end{itemize}
\item Paramètres graphique

\hspace{3cm}
\begin{tabular}{cccccccccc}
%\sffamily
%\rmfamily
 & & & &\sf Y \\
 & & & & Phi \\
 & & & &\sf H \\
 & & & &   & & &  & \\
\sf W &\sf X &\sf C &\sf V &\sf B &\sf N  \\
 &  &  &  & ech - & ech + \\
\end{tabular}
%
\end{itemize}
\begin{itemize}[leftmargin=2cm, label=\ding{32}, itemsep=0pt]%\end{itemize}
\item Caractéristique de la porteuse

\hspace{9cm}
\begin{tabular}{ccccc}
%\sffamily
%\rmfamily
\sf U &\sf I &\sf O &\sf P \\
 co -1 & co 1 & co 0 & Fréquence \\
\sf J &\sf K &\sf L &\sf M \\
  &  & carré & moinsF \\
\end{tabular}
%
\end{itemize}
%\vspace{.3cm}
Les touches de fonctions donnent un certain nombre de contrôles et d'information:
%
\begin{center}
\begin{tabular}{ccccc ccccc cc}
\multicolumn{4}{|c|}{Contrôles} & \multicolumn{4}{c}{Information} & \multicolumn{4}{|c|}{Contrôles}\\
\sf F1 &\sf F2 &\sf F3 &\sf F4 &\sf F5 &\sf F6 &\sf F7 &\sf F8 &\sf F9 &\sf F10 &\sf F11 &\sf F12 \\
\multicolumn{4}{|c|}{Équation simulé (SiCF)} & \multicolumn{4}{c}{Énergie, graphe} & \multicolumn{4}{|c|}{Vitesse de la simulation}\\
\end{tabular}
\end{center}
%
Le choix de l'équation simulée est spécifique à SiCF. {\sf F5} dresse un bilan énergétique. {\sf F6} affiche les paramètres physiques du système
\begin{center}
\begin{tabular}{cccccc}
\sf F1 &\sf F2 &\sf F3 &\sf F4 &\sf F5 &\sf F6\\
Pendules & Harmoniques & Corde & asymétrique & Énergie & Système \\
\end{tabular}
\end{center}
%
{\sf F8} permet de modifier  le graphisme de SiCP. {\sf F9} et {\sf F12} modifient rapidement le rythme de la simulation, {\sf F10} et {\sf F11} la modifie modéremment. La touche {\sf Entrée} change le mode avec ou sans attente, en mode avec attente, l'appuie sur une touche permet l'évolution du système.
\begin{center}
\begin{tabular}{cccccc}
\sf F8 &\sf F9 &\sf F10 &\sf F11 &\sf F12 & \sf Entrée \\
Support (SiCP) & -Sim & -Sim & +Sim & +Sim & mode\\
\end{tabular}
\end{center}
%\section{Détails des contrôles}
%
%\item {\bf } : \sf{} : 
%%%%%%%%%%%%%%%%%%%%%%%%%%%%%%%%%%%%%%%%%%%%%%%%%%%%%%%%%
%
%%%%%%%%%%%%%%%%%%%%%%%%%%%%%%%%%%%%%%%%%%%%%%%%%%%%%%%%%%%%%%%%%%%%%%%%%%%%%%%%%%%%%%%%%%%%%

%
%
%====================== INCLUSION DE LA BIBLIOGRAPHIE ======================
%
%récupérer les citation avec "/footnotemark" : 
\nocite{*}
%
% choix du style de la biblio
\bibliographystyle{plain}
%
% inclusion de la biblio
\cleardoublepage
\addcontentsline{toc}{chapter}{Bibliographie}
\bibliography{bibliographie.bib}
%
%====================== FIN DU DOCUMENT ======================
%
\end{document}
%%%%%%%%%%%%%%%%%%%%%%%%%%%%%%%%%%%%%%%%%%%%%%%%%%%%%%%%%%%%%%%%%%%%%%%%%%%%%%%%%
