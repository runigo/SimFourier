%
\chapter{Commande du clavier SimFourier}
%
Le clavier permet de modifier les paramètres physiques. La fenêtre graphique doit être active, le terminal affiche les informations.
%
\footnotesize
\begin{center}
\begin{tabular}{cccccccccc}
%\sffamily
%\rmfamily
\sf A &\sf Z &\sf E &\sf R &\sf T &\sf Y &\sf U &\sf I &\sf O &\sf P \\
 & env 0 & env 1 & env -1 &  &  & co -1 & co 1 & co 0 & Fréquence \\
\sf Q &\sf S &\sf D &\sf F &\sf G &\sf H &\sf J &\sf K &\sf L &\sf M \\
 &  &  &  &  &  &  &  &  &  \\
\sf W &\sf X &\sf C &\sf V &\sf B &\sf N &  &  &  & \\
 &  &  &  &  &  &  &  &  & \\
\end{tabular}
\end{center}
\vspace{.3cm}
%
\normalsize
\begin{itemize}[leftmargin=2cm, label=\ding{32}, itemsep=0pt]%\end{itemize}
\item Caractéristiques de l'enveloppe

\begin{tabular}{ccccc}
%\sffamily
%\rmfamily
\sf A &\sf Z &\sf E &\sf R &\sf T  \\
env 0 & env 1 & env -1 &  &   \\
\sf Q &\sf S &\sf D &\sf F &\sf G  \\
 &  &  &  &   \\
 & & & \sf V   & \\
 & & &  &  \\
\end{tabular}
%
\end{itemize}
\begin{itemize}[leftmargin=2cm, label=\ding{32}, itemsep=0pt]%\end{itemize}
\item Paramètres graphique

\hspace{3cm}
\begin{tabular}{cccccccccc}
%\sffamily
%\rmfamily
 & & & &\sf Y \\
 & & & & Phi \\
 & & & &\sf H \\
 & & & &   & & &  & \\
\sf W &\sf X &\sf C &\sf V &\sf B &\sf N  \\
 &  &  &  & ech - & ech + \\
\end{tabular}
%
\end{itemize}
\begin{itemize}[leftmargin=2cm, label=\ding{32}, itemsep=0pt]%\end{itemize}
\item Caractéristique de la porteuse

\hspace{9cm}
\begin{tabular}{ccccc}
%\sffamily
%\rmfamily
\sf U &\sf I &\sf O &\sf P \\
 co -1 & co 1 & co 0 & Fréquence \\
\sf J &\sf K &\sf L &\sf M \\
  &  & carré & moinsF \\
\end{tabular}
%
\end{itemize}
%\vspace{.3cm}
Les touches de fonctions donnent un certain nombre de contrôles et d'information:
%
\begin{center}
\begin{tabular}{ccccc ccccc cc}
\multicolumn{4}{|c|}{Contrôles} & \multicolumn{4}{c}{Information} & \multicolumn{4}{|c|}{Contrôles}\\
\sf F1 &\sf F2 &\sf F3 &\sf F4 &\sf F5 &\sf F6 &\sf F7 &\sf F8 &\sf F9 &\sf F10 &\sf F11 &\sf F12 \\
\multicolumn{4}{|c|}{Équation simulé (SiCF)} & \multicolumn{4}{c}{Énergie, graphe} & \multicolumn{4}{|c|}{Vitesse de la simulation}\\
\end{tabular}
\end{center}
%
Le choix de l'équation simulée est spécifique à SiCF. {\sf F5} dresse un bilan énergétique. {\sf F6} affiche les paramètres physiques du système
\begin{center}
\begin{tabular}{cccccc}
\sf F1 &\sf F2 &\sf F3 &\sf F4 &\sf F5 &\sf F6\\
Pendules & Harmoniques & Corde & asymétrique & Énergie & Système \\
\end{tabular}
\end{center}
%
{\sf F8} permet de modifier  le graphisme de SiCP. {\sf F9} et {\sf F12} modifient rapidement le rythme de la simulation, {\sf F10} et {\sf F11} la modifie modéremment. La touche {\sf Entrée} change le mode avec ou sans attente, en mode avec attente, l'appuie sur une touche permet l'évolution du système.
\begin{center}
\begin{tabular}{cccccc}
\sf F8 &\sf F9 &\sf F10 &\sf F11 &\sf F12 & \sf Entrée \\
Support (SiCP) & -Sim & -Sim & +Sim & +Sim & mode\\
\end{tabular}
\end{center}
%\section{Détails des contrôles}
%
%\item {\bf } : \sf{} : 
%%%%%%%%%%%%%%%%%%%%%%%%%%%%%%%%%%%%%%%%%%%%%%%%%%%%%%%%%
%
%%%%%%%%%%%%%%%%%%%%%%%%%%%%%%%%%%%%%%%%%%%%%%%%%%%%%%%%%%%%%%%%%%%%%%%%%%%%%%%%%%%%%%%%%%%%%
