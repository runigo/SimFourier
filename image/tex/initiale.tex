

%%%%%%%%%%%%%%%%%%%%%%%%%%%%%%                          MENU  INITIAL

\begin{scope}[xshift=-7 cm,yshift=0.0cm]

                   %  Onglets et panneau
  \begin{scope}[xshift=0 cm,yshift=20cm]
    \fill[gray,draw=gray!10!] (-0.1, 0.6) rectangle (5.1,-20.1);
    \shade[panneauOnglet]
      (2.5, 0) -- (2.7, 0.5) -- (4.8, 0.5) -- (5, 0) -- cycle;
    \draw (3.75,0.3) node [black!20!gray]{simulation};
    \fill[panneauControles] 
      (0, 0) -- (0.2, 0.5) -- (2.3, 0.5) -- (2.5, 0) -- (5, 0) -- (5, -20) -- (0, -20) -- cycle;
    \draw (1.25,0.3) node [black]{initiale};
  \end{scope}

%%%%%%%%%%%%%%%%%%%%%%%%%%%%%%      MOTIF  ENVELOPPE

  \begin{scope}[xshift=0.2 cm,yshift=14.1cm]
    \draw (2.25,5.5) node [black]{ENVELOPPE};

    \begin{scope}[xshift=3.2 cm,yshift=0.1cm] %      Grille
      \fill[boutonEteint] (-0.075,0) rectangle (1.075,4.6);
      \foreach \i in {0,1.15,...,5} {\draw[boutonEteint] (-0.075,\i)--(1.075,\i);}
      \draw[boutonEteint] (-0.075,0)--(-0.075,4.6);
      \draw[boutonEteint] (1.075,0)--(1.075,4.6);
      \begin{scope}[yshift=3.54 cm] %              Constant
        \draw[styleEteint] (0.1,0.7)--(0.9,0.7);
      \end{scope}
      \begin{scope}[yshift=2.37 cm] %              Carré NS
        \draw[styleEteint] (0.1,0.5)--(0.1,0.9)--(0.6,0.9)--(0.6,0.1)--(0.9,0.1)--(0.9,0.5);
      \end{scope}
      \begin{scope}[yshift=1.24 cm] %           Triangle NS
        \draw[styleEteint] (0.1,0.5)--(0.4,0.9)--(0.6,0.1)--(0.9,0.5);
      \end{scope}
      \begin{scope}[yshift=0.1 cm] %               Sinus
        \draw[styleEteint] (0.1,0.5) sin (0.3,0.9) cos (0.5,0.5) sin (0.7,0.1) cos (0.9,0.5);
      \end{scope}
    \end{scope}

  \end{scope}

%%%%%%%%%%%%%%%%%%%%%%%%%%%%%%  FRÉQUENCE  ENVELOPPE

  \begin{scope}[xshift=2.5 cm,yshift=16.6cm, scale=0.7]
    

%%%%%%%%%%%%%%%%%%%%%%%%%%%%%%    BOUTONS ROTATIF
%      Carrés gris
  \fill[gray!50!] (0,0) rectangle (3,3);
  \fill[gray] (1,0) rectangle (3,2);
  \fill[gray!50!black] (2,0) rectangle (3,1);
%%%%%%%%%


\newcommand{\clock}[4]{%
  \begin{scope}[xshift=2.25*#1cm,yshift=2.25*#2cm]
  % \filldraw [fill=#3, line width=1.6pt] (0,0) circle (1cm); % simple fill (unused)
  % \draw[fill] (0,0) circle (1mm); % border (unused)
  \shadedraw [inner color=#3!30!white, outer color=#3!90!black, 
    line width=1.6pt] (0,0) circle (1cm); % disk with shadow and border
  \foreach \angle in {0, 30, ..., 330} 
    \draw[line width=1pt] (\angle:0.82cm) -- (\angle:1cm);
  \foreach \angle in {0,90,180,270}
    \draw[line width=1.3pt] (\angle:0.75cm) -- (\angle:1cm);
  \draw[line width=1.6pt] (0,0) -- (90-30*#4:0.6cm); % the hand 
  \end{scope}
}
% Default on latex.ltx: \fboxsep = 3pt \fboxrule = .4pt
\fboxsep = 3pt \fboxrule = 1.5pt 
\newcounter{itimesj}
\fbox{%
\begin{tikzpicture}[scale=0.5]
\begin{scope}[line cap=round]
\foreach \i in {0,...,11}
  \foreach \j in {0,...,11} {
    % We use round by caution, because we are getting real numbers
    \pgfmathparse{round(mod(\i*\j,12))} 
    % \pgfmathresult = 3.0 gives itimesj = 3
    \pgfmathsetcounter{itimesj}{\pgfmathresult} 
    \clock{\i}{-\j}{clock\theitimesj}{\i*\j}; 
  }
\end{scope}

    \draw (-2.7,2.6) node [black]{P$_0$};
  \end{scope}
  \begin{scope}[xshift=2.5 cm,yshift=14.2cm, scale=0.7]
    

%%%%%%%%%%%%%%%%%%%%%%%%%%%%%%    BOUTONS ROTATIF
%      Carrés gris
  \fill[gray!50!] (0,0) rectangle (3,3);
  \fill[gray] (1,0) rectangle (3,2);
  \fill[gray!50!black] (2,0) rectangle (3,1);
%%%%%%%%%


\newcommand{\clock}[4]{%
  \begin{scope}[xshift=2.25*#1cm,yshift=2.25*#2cm]
  % \filldraw [fill=#3, line width=1.6pt] (0,0) circle (1cm); % simple fill (unused)
  % \draw[fill] (0,0) circle (1mm); % border (unused)
  \shadedraw [inner color=#3!30!white, outer color=#3!90!black, 
    line width=1.6pt] (0,0) circle (1cm); % disk with shadow and border
  \foreach \angle in {0, 30, ..., 330} 
    \draw[line width=1pt] (\angle:0.82cm) -- (\angle:1cm);
  \foreach \angle in {0,90,180,270}
    \draw[line width=1.3pt] (\angle:0.75cm) -- (\angle:1cm);
  \draw[line width=1.6pt] (0,0) -- (90-30*#4:0.6cm); % the hand 
  \end{scope}
}
% Default on latex.ltx: \fboxsep = 3pt \fboxrule = .4pt
\fboxsep = 3pt \fboxrule = 1.5pt 
\newcounter{itimesj}
\fbox{%
\begin{tikzpicture}[scale=0.5]
\begin{scope}[line cap=round]
\foreach \i in {0,...,11}
  \foreach \j in {0,...,11} {
    % We use round by caution, because we are getting real numbers
    \pgfmathparse{round(mod(\i*\j,12))} 
    % \pgfmathresult = 3.0 gives itimesj = 3
    \pgfmathsetcounter{itimesj}{\pgfmathresult} 
    \clock{\i}{-\j}{clock\theitimesj}{\i*\j}; 
  }
\end{scope}

    \draw (-2.7,2.6) node [black]{P$_1$};
  \end{scope}
  
%%%%%%%%%%%%%%%%%%%%%%%%%%%%%%  MOYENNE  ET  SYMÉTRIE


  \begin{scope}[xshift=2.5 cm,yshift=11.2cm, scale=0.7]
    

%%%%%%%%%%%%%%%%%%%%%%%%%%%%%%    BOUTONS ROTATIF
%      Carrés gris
  \fill[gray!50!] (0,0) rectangle (3,3);
  \fill[gray] (1,0) rectangle (3,2);
  \fill[gray!50!black] (2,0) rectangle (3,1);
%%%%%%%%%


\newcommand{\clock}[4]{%
  \begin{scope}[xshift=2.25*#1cm,yshift=2.25*#2cm]
  % \filldraw [fill=#3, line width=1.6pt] (0,0) circle (1cm); % simple fill (unused)
  % \draw[fill] (0,0) circle (1mm); % border (unused)
  \shadedraw [inner color=#3!30!white, outer color=#3!90!black, 
    line width=1.6pt] (0,0) circle (1cm); % disk with shadow and border
  \foreach \angle in {0, 30, ..., 330} 
    \draw[line width=1pt] (\angle:0.82cm) -- (\angle:1cm);
  \foreach \angle in {0,90,180,270}
    \draw[line width=1.3pt] (\angle:0.75cm) -- (\angle:1cm);
  \draw[line width=1.6pt] (0,0) -- (90-30*#4:0.6cm); % the hand 
  \end{scope}
}
% Default on latex.ltx: \fboxsep = 3pt \fboxrule = .4pt
\fboxsep = 3pt \fboxrule = 1.5pt 
\newcounter{itimesj}
\fbox{%
\begin{tikzpicture}[scale=0.5]
\begin{scope}[line cap=round]
\foreach \i in {0,...,11}
  \foreach \j in {0,...,11} {
    % We use round by caution, because we are getting real numbers
    \pgfmathparse{round(mod(\i*\j,12))} 
    % \pgfmathresult = 3.0 gives itimesj = 3
    \pgfmathsetcounter{itimesj}{\pgfmathresult} 
    \clock{\i}{-\j}{clock\theitimesj}{\i*\j}; 
  }
\end{scope}

    \draw (-2.7,2.6) node [black]{S};
  \end{scope}
  \begin{scope}[xshift=2.5 cm,yshift=8.1cm, scale=0.7]
    

%%%%%%%%%%%%%%%%%%%%%%%%%%%%%%    BOUTONS ROTATIF
%      Carrés gris
  \fill[gray!50!] (0,0) rectangle (3,3);
  \fill[gray] (1,0) rectangle (3,2);
  \fill[gray!50!black] (2,0) rectangle (3,1);
%%%%%%%%%


\newcommand{\clock}[4]{%
  \begin{scope}[xshift=2.25*#1cm,yshift=2.25*#2cm]
  % \filldraw [fill=#3, line width=1.6pt] (0,0) circle (1cm); % simple fill (unused)
  % \draw[fill] (0,0) circle (1mm); % border (unused)
  \shadedraw [inner color=#3!30!white, outer color=#3!90!black, 
    line width=1.6pt] (0,0) circle (1cm); % disk with shadow and border
  \foreach \angle in {0, 30, ..., 330} 
    \draw[line width=1pt] (\angle:0.82cm) -- (\angle:1cm);
  \foreach \angle in {0,90,180,270}
    \draw[line width=1.3pt] (\angle:0.75cm) -- (\angle:1cm);
  \draw[line width=1.6pt] (0,0) -- (90-30*#4:0.6cm); % the hand 
  \end{scope}
}
% Default on latex.ltx: \fboxsep = 3pt \fboxrule = .4pt
\fboxsep = 3pt \fboxrule = 1.5pt 
\newcounter{itimesj}
\fbox{%
\begin{tikzpicture}[scale=0.5]
\begin{scope}[line cap=round]
\foreach \i in {0,...,11}
  \foreach \j in {0,...,11} {
    % We use round by caution, because we are getting real numbers
    \pgfmathparse{round(mod(\i*\j,12))} 
    % \pgfmathresult = 3.0 gives itimesj = 3
    \pgfmathsetcounter{itimesj}{\pgfmathresult} 
    \clock{\i}{-\j}{clock\theitimesj}{\i*\j}; 
  }
\end{scope}

    \draw (-2.7,2.6) node [black]{\Large{$\phi$}};
  \end{scope}

%%%%%%%%%%%%%%%%%%%%%%%%%%%%%%    PÉRIODICITÉ,  BOSSES

  \begin{scope}[xshift=0.2 cm,yshift=8cm]
    \begin{scope}[xshift=3.2 cm,yshift=0cm]
      \fill[boutonEteint] (-0.075,0) rectangle (1.075,5.8);
      \foreach \i in {0,1.15,...,5} {\draw[boutonEteint] (-0.075,\i)--(1.075,\i);}
      \draw[boutonEteint] (-0.075,0)--(-0.075,4.6);\draw[boutonEteint] (1.075,0)--(1.075,4.6);
      \begin{scope}[yshift=4.7 cm] % 0
        \draw[styleEteint] (0.1,0.3)--(0.9,0.3);
        \draw[styleEteint] (0.5,0.3)--(0.5,0.7);
      \end{scope}
      \begin{scope}[yshift=3.5 cm] % 1
        \draw[styleEteint] (0.2,0.3)--(0.2,0.7);\draw[styleEteint] (0.5,0.3)--(0.5,0.7);
        \draw[styleEteint] (0.8,0.3)--(0.8,0.7); 
        \draw[styleEteint] (0.1,0.3)--(0.9,0.3);
      \end{scope}
      \begin{scope}[xshift=0.1 cm,yshift=2.45 cm] % Gaussienne
          \draw [styleEteint, domain=-0.4:0.4, samples=80]
            plot (\x+0.4, {0.1+0.6*exp(-100 *\x * \x)}) ;
      \end{scope}
      \begin{scope}[xshift=0.1 cm,yshift=1.33 cm] % Lorentzienne
          \draw [styleEteint, domain=-0.4:0.4, samples=80]
            plot (\x+0.4, {0.7/(1 + 100 * \x * \x)}) ;
      \end{scope}
      \begin{scope}[xshift=0.1 cm,yshift=0.3 cm] % sinc
          \draw [styleEteint, domain=-3.55:3.55, samples=80]
            plot (0.1*\x+0.4, {(1 - \x * \x * ( 0.31) + \x * \x * \x * \x * (0.05)
             - \x * \x * \x * \x * \x * \x * ( 0.0023)-0.4}) ;
      \end{scope}
    \end{scope}
  \end{scope}


%%%%%%%%%%%%%%%%%%%%%%%%%%%%%%  PORTEUSE
  \begin{scope}[xshift=0.2 cm,yshift=2cm] % style de la porteuse
    \draw (2.25,5.2) node [black]{PORTEUSE};
    \begin{scope}[xshift=3.2 cm]  %      Grille
      \fill[boutonEteint] (-0.075,0) rectangle (1.075,4.6);
      \foreach \i in {0,1.15,...,5} {\draw[boutonEteint] (-0.075,\i)--(1.075,\i);}
      \draw[boutonEteint] (-0.075,0)--(-0.075,4.6);\draw[boutonEteint] (1.075,0)--(1.075,4.6);
      \begin{scope}[yshift=3.8 cm] % Constant
          \draw[styleEteint] (0.1,0.5)--(0.9,0.5);
      \end{scope}
      \begin{scope}[yshift=2.4 cm] % dirac
          \draw[styleEteint, >=latex, ->] (0.3,0.15)--(0.3,0.85);
          \draw[styleEteint, >=latex, ->] (0.7,0.15)--(0.7,0.85);
          \draw[styleEteint] (0.1,0.15)--(0.9,0.15);
      \end{scope}
      \begin{scope}[yshift=1.25 cm] % Sinus
          \draw[styleEteint] (0.1,0.5) sin (0.3,0.9) cos (0.5,0.5) sin (0.7,0.1) cos (0.9,0.5);
      \end{scope}
  \begin{scope}[xshift=0.3 cm,yshift=0.15cm] % spirale
  \draw [styleEteint, domain=0.08:0.9, samples=80]
  plot (\x, {0.6+0.35*sin(15*\x r)}, {0.6+0.35*cos(15*\x r)}) ;
  \end{scope}
    \end{scope}
  \end{scope}
%%%%%%%%%%%%%%%%%%%%%%%%%%%%%%  FRÉQUENCE  PORTEUSE
   % \draw (2.5,6.35) node [black]{Fréquence};
%      Carrés gris
  \begin{scope}[xshift=2.5 cm,yshift=4.4cm, scale=0.7]
    

%%%%%%%%%%%%%%%%%%%%%%%%%%%%%%    BOUTONS ROTATIF
%      Carrés gris
  \fill[gray!50!] (0,0) rectangle (3,3);
  \fill[gray] (1,0) rectangle (3,2);
  \fill[gray!50!black] (2,0) rectangle (3,1);
%%%%%%%%%


\newcommand{\clock}[4]{%
  \begin{scope}[xshift=2.25*#1cm,yshift=2.25*#2cm]
  % \filldraw [fill=#3, line width=1.6pt] (0,0) circle (1cm); % simple fill (unused)
  % \draw[fill] (0,0) circle (1mm); % border (unused)
  \shadedraw [inner color=#3!30!white, outer color=#3!90!black, 
    line width=1.6pt] (0,0) circle (1cm); % disk with shadow and border
  \foreach \angle in {0, 30, ..., 330} 
    \draw[line width=1pt] (\angle:0.82cm) -- (\angle:1cm);
  \foreach \angle in {0,90,180,270}
    \draw[line width=1.3pt] (\angle:0.75cm) -- (\angle:1cm);
  \draw[line width=1.6pt] (0,0) -- (90-30*#4:0.6cm); % the hand 
  \end{scope}
}
% Default on latex.ltx: \fboxsep = 3pt \fboxrule = .4pt
\fboxsep = 3pt \fboxrule = 1.5pt 
\newcounter{itimesj}
\fbox{%
\begin{tikzpicture}[scale=0.5]
\begin{scope}[line cap=round]
\foreach \i in {0,...,11}
  \foreach \j in {0,...,11} {
    % We use round by caution, because we are getting real numbers
    \pgfmathparse{round(mod(\i*\j,12))} 
    % \pgfmathresult = 3.0 gives itimesj = 3
    \pgfmathsetcounter{itimesj}{\pgfmathresult} 
    \clock{\i}{-\j}{clock\theitimesj}{\i*\j}; 
  }
\end{scope}

    \draw (-2.7,2.6) node [black]{P$_0$};
  \end{scope}
  \begin{scope}[xshift=2.5 cm,yshift=2cm, scale=0.7]
    

%%%%%%%%%%%%%%%%%%%%%%%%%%%%%%    BOUTONS ROTATIF
%      Carrés gris
  \fill[gray!50!] (0,0) rectangle (3,3);
  \fill[gray] (1,0) rectangle (3,2);
  \fill[gray!50!black] (2,0) rectangle (3,1);
%%%%%%%%%


\newcommand{\clock}[4]{%
  \begin{scope}[xshift=2.25*#1cm,yshift=2.25*#2cm]
  % \filldraw [fill=#3, line width=1.6pt] (0,0) circle (1cm); % simple fill (unused)
  % \draw[fill] (0,0) circle (1mm); % border (unused)
  \shadedraw [inner color=#3!30!white, outer color=#3!90!black, 
    line width=1.6pt] (0,0) circle (1cm); % disk with shadow and border
  \foreach \angle in {0, 30, ..., 330} 
    \draw[line width=1pt] (\angle:0.82cm) -- (\angle:1cm);
  \foreach \angle in {0,90,180,270}
    \draw[line width=1.3pt] (\angle:0.75cm) -- (\angle:1cm);
  \draw[line width=1.6pt] (0,0) -- (90-30*#4:0.6cm); % the hand 
  \end{scope}
}
% Default on latex.ltx: \fboxsep = 3pt \fboxrule = .4pt
\fboxsep = 3pt \fboxrule = 1.5pt 
\newcounter{itimesj}
\fbox{%
\begin{tikzpicture}[scale=0.5]
\begin{scope}[line cap=round]
\foreach \i in {0,...,11}
  \foreach \j in {0,...,11} {
    % We use round by caution, because we are getting real numbers
    \pgfmathparse{round(mod(\i*\j,12))} 
    % \pgfmathresult = 3.0 gives itimesj = 3
    \pgfmathsetcounter{itimesj}{\pgfmathresult} 
    \clock{\i}{-\j}{clock\theitimesj}{\i*\j}; 
  }
\end{scope}

    \draw (-2.7,2.6) node [black]{P$_1$};
  \end{scope}

\end{scope}

