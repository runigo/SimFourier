% SimFourier
%
% Simulateur de transformation de fourier
%
\documentclass[a4paper,12pt,oneside]{article}
\usepackage[usenames, dvipsnames]{xcolor}
\usepackage{comment}
\usepackage{tikz}
\usetikzlibrary{trees}
\usetikzlibrary{decorations.pathmorphing}
\usetikzlibrary{decorations.markings}
\usetikzlibrary{decorations.pathreplacing,calligraphy}
\usetikzlibrary{decorations.pathmorphing,calc,shapes,shapes.geometric,patterns}
%\usetikzlibrary{decorations}
\usetikzlibrary{angles, quotes}
\usepackage{verbatim}
\usepackage[active,tightpage]{preview}
\usepackage[T1]{fontenc}
\usepackage[utf8]{inputenc}
%\usepackage[greek, frenchb]{babel}
\PreviewEnvironment{tikzpicture}
\setlength\PreviewBorder{0pt}%
%%%

%======================== DEFINITION COMMANDES ========================
\newcommand{\mt}[1]{\text{#1}}
\newcommand{\ul}[1]{\underline{#1}}
\newcommand{\mc}[1]{\mathcal{#1}}
\newcommand{\pt}[1]{\dot{\text{#1}}}
%======================== DEBUT DU DOCUMENT ========================
\begin{document}
\def\scl{1}
\sffamily

\begin{tikzpicture}[
  scale=\scl,
  %panneauControles/.style={black!10!brown!10!,draw=black!60!brown!40!,thick},
  panneauControles/.style={black!10!brown!10!},
  panneauOnglet/.style={bottom color=black!60!brown!60!, top color=black!20!brown!20!},
  boutonEteint/.style={black!20!brown!20!YellowOrange!20!,draw=black!30!brown!40!,very thick},
  %boutonEteint/.style={black!50!brown!50!,draw=black!30!brown!40!,very thick},
  styleEteint/.style={Bittersweet!50!YellowOrange,draw=black,very thick},
  %boutonSelect/.style={white,draw=black,very thick},
  %bouton/.style={inner color=Yellow, outer color=YellowOrange},
  bouton/.style={inner color=Bittersweet!50!YellowOrange, outer color=Bittersweet!50!YellowOrange},
  traitEteint/.style={black!30!,draw=black!10!,thick},
  traitSelect/.style={black!70!,draw=black!30!,thick},
  petitbouton/.style={white,draw=black, thick}]

%
%

%%%%%%%%%%%%%%%%%%%%%%%%%%%%%%                          MENU  SIMULATION
\begin{scope}[xshift=7 cm,yshift=0.0cm]

                   %  Onglet
  \begin{scope}[xshift=0 cm,yshift=20cm]
    \draw[gray!10!] (-0.1, 0.6) rectangle (5.1,-20.1);
    \shade[panneauOnglet]
      (0, 0) -- (0.2, 0.5) -- (2.3, 0.5) -- (2.5, 0) -- (5, 0) -- cycle;
    \draw (1.25,0.3) node [black!20!gray]{initiale};
    \fill[panneauControles] 
      (2.5, 0) -- (2.7, 0.5) -- (4.8, 0.5) -- (5, 0) -- (5, -20) -- (0, -20) -- (0, 0) -- cycle;
    \draw (3.75,0.3) node [black]{simulation};
  \end{scope}

%%%%%%%%%%%%%%%%%%%%%%%%%%%%%%%%%%%%%        COUPLAGE
\def\diametre{0.3}
  \begin{scope}[xshift=0.5 cm,yshift=16.4cm]
    \fill[petitbouton] (3.5,-1.9) circle (\diametre);
    \node[scale=\scl] at (3.5,-1.20) {Mixte};
    \fill[petitbouton] (3.5,-0.45) circle (\diametre);
    \node[scale=\scl] at (3.5,0.15) {Fixe};
    \fill[petitbouton] (3.5,1.0) circle (\diametre);
    \node[scale=\scl] at (3.5,1.6) {Libre};
    \fill[petitbouton] (3.5,2.35) circle (\diametre);
    \node[scale=\scl] at (3.5,2.95) {\footnotesize{Périodique}};
  \end{scope}

    \draw (1.2,19.5) node [black]{Couplage};
  \begin{scope}[xshift=0.4 cm,yshift=17.1cm, scale=0.7]


%%%%%%%%%%%%%%%%%%%%%%%%%%%%%%    BOUTONS ROTATIF
%      Carrés gris
  \fill[gray!50!] (0,0) rectangle (3,3);
  \fill[gray] (1,0) rectangle (3,2);
  \fill[gray!50!black] (2,0) rectangle (3,1);
%%%%%%%%%


\newcommand{\clock}[4]{%
  \begin{scope}[xshift=2.25*#1cm,yshift=2.25*#2cm]
  % \filldraw [fill=#3, line width=1.6pt] (0,0) circle (1cm); % simple fill (unused)
  % \draw[fill] (0,0) circle (1mm); % border (unused)
  \shadedraw [inner color=#3!30!white, outer color=#3!90!black, 
    line width=1.6pt] (0,0) circle (1cm); % disk with shadow and border
  \foreach \angle in {0, 30, ..., 330} 
    \draw[line width=1pt] (\angle:0.82cm) -- (\angle:1cm);
  \foreach \angle in {0,90,180,270}
    \draw[line width=1.3pt] (\angle:0.75cm) -- (\angle:1cm);
  \draw[line width=1.6pt] (0,0) -- (90-30*#4:0.6cm); % the hand 
  \end{scope}
}
% Default on latex.ltx: \fboxsep = 3pt \fboxrule = .4pt
\fboxsep = 3pt \fboxrule = 1.5pt 
\newcounter{itimesj}
\fbox{%
\begin{tikzpicture}[scale=0.5]
\begin{scope}[line cap=round]
\foreach \i in {0,...,11}
  \foreach \j in {0,...,11} {
    % We use round by caution, because we are getting real numbers
    \pgfmathparse{round(mod(\i*\j,12))} 
    % \pgfmathresult = 3.0 gives itimesj = 3
    \pgfmathsetcounter{itimesj}{\pgfmathresult} 
    \clock{\i}{-\j}{clock\theitimesj}{\i*\j}; 
  }
\end{scope}

  \end{scope}

    \draw (1.2,16.4) node [black]{Masse};
  \begin{scope}[xshift=0.4 cm,yshift=14cm, scale=0.7]


%%%%%%%%%%%%%%%%%%%%%%%%%%%%%%    BOUTONS ROTATIF
%      Carrés gris
  \fill[gray!50!] (0,0) rectangle (3,3);
  \fill[gray] (1,0) rectangle (3,2);
  \fill[gray!50!black] (2,0) rectangle (3,1);
%%%%%%%%%


\newcommand{\clock}[4]{%
  \begin{scope}[xshift=2.25*#1cm,yshift=2.25*#2cm]
  % \filldraw [fill=#3, line width=1.6pt] (0,0) circle (1cm); % simple fill (unused)
  % \draw[fill] (0,0) circle (1mm); % border (unused)
  \shadedraw [inner color=#3!30!white, outer color=#3!90!black, 
    line width=1.6pt] (0,0) circle (1cm); % disk with shadow and border
  \foreach \angle in {0, 30, ..., 330} 
    \draw[line width=1pt] (\angle:0.82cm) -- (\angle:1cm);
  \foreach \angle in {0,90,180,270}
    \draw[line width=1.3pt] (\angle:0.75cm) -- (\angle:1cm);
  \draw[line width=1.6pt] (0,0) -- (90-30*#4:0.6cm); % the hand 
  \end{scope}
}
% Default on latex.ltx: \fboxsep = 3pt \fboxrule = .4pt
\fboxsep = 3pt \fboxrule = 1.5pt 
\newcounter{itimesj}
\fbox{%
\begin{tikzpicture}[scale=0.5]
\begin{scope}[line cap=round]
\foreach \i in {0,...,11}
  \foreach \j in {0,...,11} {
    % We use round by caution, because we are getting real numbers
    \pgfmathparse{round(mod(\i*\j,12))} 
    % \pgfmathresult = 3.0 gives itimesj = 3
    \pgfmathsetcounter{itimesj}{\pgfmathresult} 
    \clock{\i}{-\j}{clock\theitimesj}{\i*\j}; 
  }
\end{scope}

  \end{scope}

%%%%%%%%%%%%%%%%%%%%%%%%%%%%%%%%%%%%%        DISSIPATION
  \begin{scope}[xshift=0.5 cm,yshift=9.4cm]
    \fill[petitbouton] (3.5,-0.45) circle (\diametre);
    \node[scale=\scl] at (3.5,0.15) {\footnotesize{Extrémité}};
    \fill[petitbouton] (3.5,1.0) circle (\diametre);
    \node[scale=\scl] at (3.5,1.6) {Nulle};
    \fill[petitbouton] (3.5,2.35) circle (\diametre);
    \node[scale=\scl] at (3.5,2.95) {\footnotesize{Uniforme}};
  \end{scope}

    \draw (1.4,11.8) node [black]{Dissipation};
  \begin{scope}[xshift=0.4 cm,yshift=9.3cm, scale=0.7]


%%%%%%%%%%%%%%%%%%%%%%%%%%%%%%    BOUTONS ROTATIF
%      Carrés gris
  \fill[gray!50!] (0,0) rectangle (3,3);
  \fill[gray] (1,0) rectangle (3,2);
  \fill[gray!50!black] (2,0) rectangle (3,1);
%%%%%%%%%


\newcommand{\clock}[4]{%
  \begin{scope}[xshift=2.25*#1cm,yshift=2.25*#2cm]
  % \filldraw [fill=#3, line width=1.6pt] (0,0) circle (1cm); % simple fill (unused)
  % \draw[fill] (0,0) circle (1mm); % border (unused)
  \shadedraw [inner color=#3!30!white, outer color=#3!90!black, 
    line width=1.6pt] (0,0) circle (1cm); % disk with shadow and border
  \foreach \angle in {0, 30, ..., 330} 
    \draw[line width=1pt] (\angle:0.82cm) -- (\angle:1cm);
  \foreach \angle in {0,90,180,270}
    \draw[line width=1.3pt] (\angle:0.75cm) -- (\angle:1cm);
  \draw[line width=1.6pt] (0,0) -- (90-30*#4:0.6cm); % the hand 
  \end{scope}
}
% Default on latex.ltx: \fboxsep = 3pt \fboxrule = .4pt
\fboxsep = 3pt \fboxrule = 1.5pt 
\newcounter{itimesj}
\fbox{%
\begin{tikzpicture}[scale=0.5]
\begin{scope}[line cap=round]
\foreach \i in {0,...,11}
  \foreach \j in {0,...,11} {
    % We use round by caution, because we are getting real numbers
    \pgfmathparse{round(mod(\i*\j,12))} 
    % \pgfmathresult = 3.0 gives itimesj = 3
    \pgfmathsetcounter{itimesj}{\pgfmathresult} 
    \clock{\i}{-\j}{clock\theitimesj}{\i*\j}; 
  }
\end{scope}

  \end{scope}

%%%%%%%%%%%%%%%%%%%%%%%%%%%%%%%%%%%%%        MOTEUR
  \begin{scope}[xshift=0.5 cm,yshift=5.4cm]
    \fill[petitbouton] (3.5,-1.9) circle (\diametre);
    \node[scale=\scl] at (3.5,-1.20) {Mixte};
    \fill[petitbouton] (3.5,-0.45) circle (\diametre);
    \node[scale=\scl] at (3.5,0.15) {Fixe};
    \fill[petitbouton] (3.5,1.0) circle (\diametre);
    \node[scale=\scl] at (3.5,1.6) {Libre};
    \fill[petitbouton] (3.5,2.35) circle (\diametre);
    \node[scale=\scl] at (3.5,2.95) {\footnotesize{Périodique}};
  \end{scope}

  \begin{scope}[xshift=0.4 cm,yshift=5.1cm, scale=0.7]


%%%%%%%%%%%%%%%%%%%%%%%%%%%%%%    BOUTONS ROTATIF
%      Carrés gris
  \fill[gray!50!] (0,0) rectangle (3,3);
  \fill[gray] (1,0) rectangle (3,2);
  \fill[gray!50!black] (2,0) rectangle (3,1);
%%%%%%%%%


\newcommand{\clock}[4]{%
  \begin{scope}[xshift=2.25*#1cm,yshift=2.25*#2cm]
  % \filldraw [fill=#3, line width=1.6pt] (0,0) circle (1cm); % simple fill (unused)
  % \draw[fill] (0,0) circle (1mm); % border (unused)
  \shadedraw [inner color=#3!30!white, outer color=#3!90!black, 
    line width=1.6pt] (0,0) circle (1cm); % disk with shadow and border
  \foreach \angle in {0, 30, ..., 330} 
    \draw[line width=1pt] (\angle:0.82cm) -- (\angle:1cm);
  \foreach \angle in {0,90,180,270}
    \draw[line width=1.3pt] (\angle:0.75cm) -- (\angle:1cm);
  \draw[line width=1.6pt] (0,0) -- (90-30*#4:0.6cm); % the hand 
  \end{scope}
}
% Default on latex.ltx: \fboxsep = 3pt \fboxrule = .4pt
\fboxsep = 3pt \fboxrule = 1.5pt 
\newcounter{itimesj}
\fbox{%
\begin{tikzpicture}[scale=0.5]
\begin{scope}[line cap=round]
\foreach \i in {0,...,11}
  \foreach \j in {0,...,11} {
    % We use round by caution, because we are getting real numbers
    \pgfmathparse{round(mod(\i*\j,12))} 
    % \pgfmathresult = 3.0 gives itimesj = 3
    \pgfmathsetcounter{itimesj}{\pgfmathresult} 
    \clock{\i}{-\j}{clock\theitimesj}{\i*\j}; 
  }
\end{scope}

    \draw (1.2,2.5) node [black]{Couplage};
  \end{scope}

  \begin{scope}[xshift=0.4 cm,yshift=2cm, scale=0.7]


%%%%%%%%%%%%%%%%%%%%%%%%%%%%%%    BOUTONS ROTATIF
%      Carrés gris
  \fill[gray!50!] (0,0) rectangle (3,3);
  \fill[gray] (1,0) rectangle (3,2);
  \fill[gray!50!black] (2,0) rectangle (3,1);
%%%%%%%%%


\newcommand{\clock}[4]{%
  \begin{scope}[xshift=2.25*#1cm,yshift=2.25*#2cm]
  % \filldraw [fill=#3, line width=1.6pt] (0,0) circle (1cm); % simple fill (unused)
  % \draw[fill] (0,0) circle (1mm); % border (unused)
  \shadedraw [inner color=#3!30!white, outer color=#3!90!black, 
    line width=1.6pt] (0,0) circle (1cm); % disk with shadow and border
  \foreach \angle in {0, 30, ..., 330} 
    \draw[line width=1pt] (\angle:0.82cm) -- (\angle:1cm);
  \foreach \angle in {0,90,180,270}
    \draw[line width=1.3pt] (\angle:0.75cm) -- (\angle:1cm);
  \draw[line width=1.6pt] (0,0) -- (90-30*#4:0.6cm); % the hand 
  \end{scope}
}
% Default on latex.ltx: \fboxsep = 3pt \fboxrule = .4pt
\fboxsep = 3pt \fboxrule = 1.5pt 
\newcounter{itimesj}
\fbox{%
\begin{tikzpicture}[scale=0.5]
\begin{scope}[line cap=round]
\foreach \i in {0,...,11}
  \foreach \j in {0,...,11} {
    % We use round by caution, because we are getting real numbers
    \pgfmathparse{round(mod(\i*\j,12))} 
    % \pgfmathresult = 3.0 gives itimesj = 3
    \pgfmathsetcounter{itimesj}{\pgfmathresult} 
    \clock{\i}{-\j}{clock\theitimesj}{\i*\j}; 
  }
\end{scope}

    \draw (1.2,2.4) node [black]{Masse};
  \end{scope}





\end{scope}



%
%


%%%%%%%%%%%%%%%%%%%%%%%%%%%%%%                          MENU  INITIAL
  \begin{scope}[xshift=-7 cm,yshift=0.0cm]

                   %  Onglets et panneau
  \begin{scope}[xshift=0 cm,yshift=20cm]
    \draw[gray!10!] (-0.1, 0.6) rectangle (5.1,-20.1); % \draw[gray!10!] (0, 0.5) rectangle (5,-20);
    \shade[panneauOnglet]
      (2.5, 0) -- (2.7, 0.5) -- (4.8, 0.5) -- (5, 0) -- cycle;
    \draw (3.75,0.3) node [black!20!gray]{simulation};
    \fill[panneauControles] 
      (0, 0) -- (0.2, 0.5) -- (2.3, 0.5) -- (2.5, 0) -- (5, 0) -- (5, -20) -- (0, -20) -- cycle;
    \draw (1.25,0.3) node [black]{initiale};
  \end{scope}

                   %  Motif
  \begin{scope}[xshift=0.2 cm,yshift=18cm] % style du motif
    \draw (2.25,1.4) node [black]{ENVELOPPE};
    \begin{scope}[xshift=0 cm,yshift=0cm] %      Grille
        \fill[boutonEteint] (0,-0.075) rectangle (4.6,1.075);
        \foreach \i in {0,1.15,...,5} {\draw[boutonEteint] (\i,-0.075)--(\i,1.075);}
        \draw[boutonEteint] (0,-0.075)--(4.6,-0.075);\draw[boutonEteint] (0,1.075)--(4.6,1.075);
    \end{scope}

    \begin{scope}[xshift=0.1 cm] % Constant
      \draw[styleEteint] (0.1,0.7)--(0.9,0.7);
    \end{scope}
    \begin{scope}[xshift=1.24 cm] % Carré NS  --(0.2,0.5)  --(0.8,0.5)
      \draw[styleEteint] (0.1,0.5)--(0.1,0.9)--(0.6,0.9)--(0.6,0.1)--(0.9,0.1)--(0.9,0.5);
    \end{scope}
    \begin{scope}[xshift=2.37 cm] % Triangle NS
      \draw[styleEteint] (0.1,0.5)--(0.4,0.9)--(0.6,0.1)--(0.9,0.5);
    \end{scope}
    \begin{scope}[xshift=3.54 cm] % Sinus
      \draw[styleEteint] (0.1,0.5) sin (0.3,0.9) cos (0.5,0.5) sin (0.7,0.1) cos (0.9,0.5);
    \end{scope}
  \end{scope}
                   %  Fréquence envellope
    \draw (2.5,17.35) node [black]{Largeur};
%      Carrés gris
  \begin{scope}[xshift=0.2 cm,yshift=15cm, scale=0.7]
  \fill[gray!50!] (0,0) rectangle (3,3);
  \fill[gray] (1.3,0) rectangle (3,1.7);
  \end{scope}
  \begin{scope}[xshift=2.7 cm,yshift=15cm, scale=0.7]
  \fill[gray!50!] (0,0) rectangle (3,3);
  \fill[gray] (1.3,0) rectangle (3,1.7);
  \end{scope}
                   %  Moyenne et symetrie
    \draw (2.5,14.35) node [black]{Symétrie};

%      Carrés gris
  \begin{scope}[xshift=0.2 cm,yshift=12cm, scale=0.7]
  \fill[gray!50!] (0,0) rectangle (3,3);
  \fill[gray] (1.3,0) rectangle (3,1.7);
  \end{scope}
  \begin{scope}[xshift=2.7 cm,yshift=12cm, scale=0.7]
  \fill[gray!50!] (0,0) rectangle (3,3);
  \fill[gray] (1.3,0) rectangle (3,1.7);
  \end{scope}


                   %  Périodicité / bosse
  \begin{scope}[xshift=0.2 cm,yshift=10cm] % style du motif
    \draw (2.25,1.4) node [black]{Périodicité};
    \begin{scope}[xshift=0 cm,yshift=0cm] %      Grille
        \fill[boutonEteint] (0,-0.075) rectangle (4.6,1.075);
        \foreach \i in {0,1.15,...,5} {\draw[boutonEteint] (\i,-0.075)--(\i,1.075);}
        \draw[boutonEteint] (0,-0.075)--(4.6,-0.075);\draw[boutonEteint] (0,1.075)--(4.6,1.075);
    \end{scope}

    \begin{scope}[xshift=0.1 cm] % 0
      \draw[styleEteint] (0.1,0.3)--(0.9,0.3);
      \draw[styleEteint] (0.5,0.3)--(0.5,0.7);
    \end{scope}
    \begin{scope}[xshift=1.24 cm] % 1
      \draw[styleEteint] (0.2,0.3)--(0.2,0.7);\draw[styleEteint] (0.5,0.3)--(0.5,0.7);
      \draw[styleEteint] (0.8,0.3)--(0.8,0.7); 
      \draw[styleEteint] (0.1,0.3)--(0.9,0.3);
    \end{scope}
    \begin{scope}[xshift=2.45 cm] % Gaussienne
        \draw [styleEteint, domain=-0.4:0.4, samples=80]
          plot (\x+0.4, {0.1+0.6*exp(-100 *\x * \x)}) ;
    \end{scope}
    \begin{scope}[xshift=3.6 cm] % Lorentzienne
        \draw [styleEteint, domain=-0.4:0.4, samples=80]
          plot (\x+0.4, {0.7/(1 + 100 * \x * \x)}) ;
    \end{scope}
  \end{scope}


                   %  Porteuse
  \begin{scope}[xshift=0.2 cm,yshift=3cm] % style de la porteuse
    \draw (2.25,5.4) node [black]{PORTEUSE};
    \begin{scope}[xshift=0 cm,yshift=4cm]  %      Grille
        \fill[boutonEteint] (0,-0.075) rectangle (4.6,1.075);
        \foreach \i in {0,1.15,...,5} {\draw[boutonEteint] (\i,-0.075)--(\i,1.075);}
        \draw[boutonEteint] (0,-0.075)--(4.6,-0.075);\draw[boutonEteint] (0,1.075)--(4.6,1.075);
    \end{scope}
    \begin{scope}[xshift=0.1 cm,yshift=4cm] % Constant
        \draw[styleEteint] (0.1,0.5)--(0.9,0.5);
    \end{scope}
    \begin{scope}[xshift=1.25 cm,yshift=4cm] % dirac
        \draw[styleEteint, >=latex, ->] (0.3,0.15)--(0.3,0.85); \draw[styleEteint, >=latex, ->] (0.7,0.15)--(0.7,0.85);
        \draw[styleEteint] (0.1,0.15)--(0.9,0.15);
    \end{scope}
    \begin{scope}[xshift=2.4 cm,yshift=4cm] % Sinus
        \draw[styleEteint] (0.1,0.5) sin (0.3,0.9) cos (0.5,0.5) sin (0.7,0.1) cos (0.9,0.5);
    \end{scope}
    \begin{scope}[xshift=3.8 cm,yshift=4.15cm] % spirale
        \draw [styleEteint, domain=0.08:0.9, samples=80]
          plot (\x, {0.6+0.35*sin(15*\x r)}, {0.6+0.35*cos(15*\x r)}) ;
    \end{scope}
  \end{scope}
                   %  Fréquence Porteuse
    \draw (2.5,6.35) node [black]{Fréquence};
%      Carrés gris
  \begin{scope}[xshift=0.2 cm,yshift=4cm, scale=0.7]
  \fill[gray!50!] (0,0) rectangle (3,3);
  \fill[gray] (1.3,0) rectangle (3,1.7);
  \end{scope}
  \begin{scope}[xshift=2.7 cm,yshift=4cm, scale=0.7]
  \fill[gray!50!] (0,0) rectangle (3,3);
  \fill[gray] (1.3,0) rectangle (3,1.7);
  \end{scope}

  \end{scope}


%
%

%%%%%%%%%%%%%%%%%%%%%%%%%%%%%%             MENU  FILTRAGE  SELECTIFS

                   %  Panneau
  \begin{scope}[xshift=0 cm,yshift=0cm]
    \fill[gray,draw=gray!10!] (-6.2, -0.2) rectangle (8.2,0.9);
    \fill[panneauControles]
      (-6.1, -0.1) rectangle (8.1, 0.8);
  \end{scope}

%%%%%%%%%%%%%%%%%%%%%%%%%%%%%%      PASSE  BAS

  \begin{scope}[xshift=-7 cm,yshift=0cm, scale=0.7]
    \begin{scope}[xshift=2 cm,yshift=0cm] %      Grille
      \fill[boutonEteint] (0.1,0.0) rectangle (4.6,1);
      \foreach \i in {1.1, 2.3,...,4} {\draw[boutonEteint] (\i,-0.0)--(\i,1.0);}
      \begin{scope}[xshift=0 cm] %             éteint
        \draw (0.6,0.5) node [gray]{\bf{1}};
      \end{scope}
      \begin{scope}[xshift=1.2 cm] %              Symétrique
        \draw[styleEteint] (0.1,0.3)--(0.3,0.3)--(0.3,0.7)--(0.7,0.7)--(0.7,0.3)--(0.9,0.3);
      \end{scope}
      \begin{scope}[xshift=2.4 cm] %              Dissymétrie 1
        \draw[styleEteint] (0.1,0.3)--(0.3,0.3)--(0.3,0.7)--(0.9,0.7);
      \end{scope}
      \begin{scope}[xshift=3.6 cm] %              Dissymétrie 2
        \draw[styleEteint] (0.1,0.7)--(0.7,0.7)--(0.7,0.3)--(0.9,0.3);
      \end{scope}
    \end{scope}
  \end{scope}


%%%%%%%%%%%%%%%%%%%%%%%%%%%%%%      PASSE  HAUT

  \begin{scope}[xshift=-2.5 cm,yshift=0cm, scale=0.7]
    \begin{scope}[xshift=2 cm,yshift=0cm] %      Grille
      \fill[boutonEteint] (0.1,0.0) rectangle (4.6,1);
      \foreach \i in {1.1, 2.3,...,4} {\draw[boutonEteint] (\i,-0.0)--(\i,1.0);}
      \begin{scope}[xshift=0 cm] %             éteint
        \draw (0.6,0.5) node [gray]{\bf{2}};
      \end{scope}
      \begin{scope}[xshift=1.2 cm] %              Symétrique
        \draw[styleEteint] (0.1,0.7)--(0.3,0.7)--(0.3,0.3)--(0.7,0.3)--(0.7,0.7)--(0.9,0.7);
      \end{scope}
      \begin{scope}[xshift=2.4 cm] %              Dissymétrie 1
        \draw[styleEteint] (0.1,0.7)--(0.3,0.7)--(0.3,0.3)--((0.9,0.3);
      \end{scope}
      \begin{scope}[xshift=3.6 cm] %              Dissymétrie 2
        \draw[styleEteint] (0.1,0.3)--(0.7,0.3)--(0.7,0.7)--(0.9,0.7);
      \end{scope}
    \end{scope}
  \end{scope}


%%%%%%%%%%%%%%%%%%%%%%%%%%%%%%      PASSE  BANDE

  \begin{scope}[xshift=2 cm,yshift=0cm, scale=0.7]
    \begin{scope}[xshift=2 cm,yshift=0cm] %      Grille
      \fill[boutonEteint] (0.1,0.0) rectangle (5.8,1);
      \foreach \i in {1.1, 2.3,...,5} {\draw[boutonEteint] (\i,-0.0)--(\i,1.0);}
      \begin{scope}[xshift=0 cm] %             éteint
        \draw (0.6,0.5) node [gray]{\bf{3}};
      \end{scope}
      \begin{scope}[xshift=1.2 cm] %              Symétrique
        \draw[styleEteint] (0.1,0.3)--(0.2,0.3)--(0.2,0.7)--(0.3,0.7)--(0.3,0.3)--(0.7,0.3)
        --(0.7,0.7)--(0.8,0.7)--(0.8,0.3)--(0.9,0.3);
      \end{scope}
      \begin{scope}[xshift=2.4 cm] %              Dissymétrie 1
        \draw[styleEteint] (0.1,0.3)--(0.2,0.3)--(0.2,0.7)--(0.3,0.7)--(0.3,0.3)--(0.9,0.3);
      \end{scope}
      \begin{scope}[xshift=3.6 cm] %              Dissymétrie 2
        \draw[styleEteint] (0.1,0.3)--(0.7,0.3)--(0.7,0.7)--(0.8,0.7)--(0.8,0.3)--(0.9,0.3);
      \end{scope}
      \begin{scope}[xshift=4.8 cm] %              Inverse
        \draw (0.5,0.5) node [gray]{inv};
      \end{scope}
    \end{scope}
  \end{scope}

%%%%%%%%%%%%%%%%%%%%%%%%%%%%%%      PASSE  BAS  ACTIVÉ

  \begin{scope}[xshift=-7 cm,yshift=1.3cm, scale=0.7]
    \begin{scope}[xshift=2 cm,yshift=0cm] %      Grille
      \fill[boutonSelect] (0.1,0.0) rectangle (4.6,1);
      \foreach \i in {1.1, 2.3,...,4} {\draw[boutonSelect] (\i,-0.0)--(\i,1.0);}
      \begin{scope}[xshift=0 cm] %             éteint
        \draw (0.6,0.5) node [gray]{\bf{1}};
      \end{scope}
      \begin{scope}[xshift=1.2 cm] %              Symétrique
        \draw[boutonSelect] (0.1,0.3)--(0.3,0.3)--(0.3,0.7)--(0.7,0.7)--(0.7,0.3)--(0.9,0.3);
      \end{scope}
      \begin{scope}[xshift=2.4 cm] %              Dissymétrie 1
        \draw[boutonSelect] (0.1,0.3)--(0.3,0.3)--(0.3,0.7)--(0.9,0.7);
      \end{scope}
      \begin{scope}[xshift=3.6 cm] %              Dissymétrie 2
        \draw[boutonSelect] (0.1,0.7)--(0.7,0.7)--(0.7,0.3)--(0.9,0.3);
      \end{scope}
    \end{scope}
  \end{scope}


%%%%%%%%%%%%%%%%%%%%%%%%%%%%%%      PASSE  HAUT

  \begin{scope}[xshift=-2.5 cm,yshift=1.3cm, scale=0.7]
    \begin{scope}[xshift=2 cm,yshift=0cm] %      Grille
      \fill[boutonSelect] (0.1,0.0) rectangle (4.6,1);
      \foreach \i in {1.1, 2.3,...,4} {\draw[boutonSelect] (\i,-0.0)--(\i,1.0);}
      \begin{scope}[xshift=0 cm] %             éteint
        \draw (0.6,0.5) node [gray]{\bf{2}};
      \end{scope}
      \begin{scope}[xshift=1.2 cm] %              Symétrique
        \draw[boutonSelect] (0.1,0.7)--(0.3,0.7)--(0.3,0.3)--(0.7,0.3)--(0.7,0.7)--(0.9,0.7);
      \end{scope}
      \begin{scope}[xshift=2.4 cm] %              Dissymétrie 1
        \draw[boutonSelect] (0.1,0.7)--(0.3,0.7)--(0.3,0.3)--((0.9,0.3);
      \end{scope}
      \begin{scope}[xshift=3.6 cm] %              Dissymétrie 2
        \draw[boutonSelect] (0.1,0.3)--(0.7,0.3)--(0.7,0.7)--(0.9,0.7);
      \end{scope}
    \end{scope}
  \end{scope}


%%%%%%%%%%%%%%%%%%%%%%%%%%%%%%      PASSE  BANDE

  \begin{scope}[xshift=2 cm,yshift=1.3cm, scale=0.7]
    \begin{scope}[xshift=2 cm,yshift=0cm] %      Grille
      \fill[boutonSelect] (0.1,0.0) rectangle (5.8,1);
      \foreach \i in {1.1, 2.3,...,5} {\draw[boutonSelect] (\i,-0.0)--(\i,1.0);}
      \begin{scope}[xshift=0 cm] %             éteint
        \draw (0.6,0.5) node [gray]{\bf{3}};
      \end{scope}
      \begin{scope}[xshift=1.2 cm] %              Symétrique
        \draw[boutonSelect] (0.1,0.3)--(0.2,0.3)--(0.2,0.7)--(0.3,0.7)--(0.3,0.3)--(0.7,0.3)
        --(0.7,0.7)--(0.8,0.7)--(0.8,0.3)--(0.9,0.3);
      \end{scope}
      \begin{scope}[xshift=2.4 cm] %              Dissymétrie 1
        \draw[boutonSelect] (0.1,0.3)--(0.2,0.3)--(0.2,0.7)--(0.3,0.7)--(0.3,0.3)--(0.9,0.3);
      \end{scope}
      \begin{scope}[xshift=3.6 cm] %              Dissymétrie 2
        \draw[boutonSelect] (0.1,0.3)--(0.7,0.3)--(0.7,0.7)--(0.8,0.7)--(0.8,0.3)--(0.9,0.3);
      \end{scope}
      \begin{scope}[xshift=4.8 cm] %              Inverse
        \draw (0.5,0.5) node {inv};
      \end{scope}
    \end{scope}
  \end{scope}

%%%%%%%%%%%%%%%%%%%%%%%%%%%%%%                          MENU  FILTRAGE  ROTATOIRE

\begin{scope}[xshift=-7 cm,yshift=0.0cm]

                   %  Onglets et panneau
  \begin{scope}[xshift=0 cm,yshift=19cm]
    \fill[gray,draw=gray!10!] (-0.0, 1.5) rectangle (2.2,-16.6);
    \fill[panneauControles]
      (0.1, 0.5) -- (2.1, 0.5) -- (2.1, -16.5) -- (0.1, -16.5) -- cycle;
    \fill[panneauControles]
      (0.1, 1.4) -- (2.1, 1.4) -- (2.1, 0.6) -- (0.1, 0.6) -- cycle;
    \draw (1.1,1.0) node [gray]{\bf{F 0/1}};
  \end{scope}

  \begin{scope}[xshift=3 cm,yshift=19cm]
    \fill[gray,draw=gray!10!] (0, 1.6) rectangle (2.2,0.6);
    \fill[panneauControles]
      (0.1, 1.5) -- (2.1, 1.5) -- (2.1, 0.7) -- (0.1, 0.7) -- cycle;
    \draw (1.1,1.1) node [gray]{\bf{F 0/1}};
  \end{scope}

%%%%%%%%%%%%%%%%%%%%%%%%%%%%%%      

%%%%%%%%%%%%%%%%%%%%%%%%%%%%%%  FILTRE 1

  \begin{scope}[xshift=1.9 cm,yshift=17.8cm, scale=0.5]
    

%%%%%%%%%%%%%%%%%%%%%%%%%%%%%%    BOUTONS ROTATIF
%      Carrés gris
  \fill[gray!50!] (0,0) rectangle (3,3);
  \fill[gray] (1,0) rectangle (3,2);
  \fill[gray!50!black] (2,0) rectangle (3,1);
%%%%%%%%%


\newcommand{\clock}[4]{%
  \begin{scope}[xshift=2.25*#1cm,yshift=2.25*#2cm]
  % \filldraw [fill=#3, line width=1.6pt] (0,0) circle (1cm); % simple fill (unused)
  % \draw[fill] (0,0) circle (1mm); % border (unused)
  \shadedraw [inner color=#3!30!white, outer color=#3!90!black, 
    line width=1.6pt] (0,0) circle (1cm); % disk with shadow and border
  \foreach \angle in {0, 30, ..., 330} 
    \draw[line width=1pt] (\angle:0.82cm) -- (\angle:1cm);
  \foreach \angle in {0,90,180,270}
    \draw[line width=1.3pt] (\angle:0.75cm) -- (\angle:1cm);
  \draw[line width=1.6pt] (0,0) -- (90-30*#4:0.6cm); % the hand 
  \end{scope}
}
% Default on latex.ltx: \fboxsep = 3pt \fboxrule = .4pt
\fboxsep = 3pt \fboxrule = 1.5pt 
\newcounter{itimesj}
\fbox{%
\begin{tikzpicture}[scale=0.5]
\begin{scope}[line cap=round]
\foreach \i in {0,...,11}
  \foreach \j in {0,...,11} {
    % We use round by caution, because we are getting real numbers
    \pgfmathparse{round(mod(\i*\j,12))} 
    % \pgfmathresult = 3.0 gives itimesj = 3
    \pgfmathsetcounter{itimesj}{\pgfmathresult} 
    \clock{\i}{-\j}{clock\theitimesj}{\i*\j}; 
  }
\end{scope}

    \draw (-2.7,2.6) node [black]{f$_1$};
  \end{scope}
  \begin{scope}[xshift=1.9 cm,yshift=16cm, scale=0.5]
    

%%%%%%%%%%%%%%%%%%%%%%%%%%%%%%    BOUTONS ROTATIF
%      Carrés gris
  \fill[gray!50!] (0,0) rectangle (3,3);
  \fill[gray] (1,0) rectangle (3,2);
  \fill[gray!50!black] (2,0) rectangle (3,1);
%%%%%%%%%


\newcommand{\clock}[4]{%
  \begin{scope}[xshift=2.25*#1cm,yshift=2.25*#2cm]
  % \filldraw [fill=#3, line width=1.6pt] (0,0) circle (1cm); % simple fill (unused)
  % \draw[fill] (0,0) circle (1mm); % border (unused)
  \shadedraw [inner color=#3!30!white, outer color=#3!90!black, 
    line width=1.6pt] (0,0) circle (1cm); % disk with shadow and border
  \foreach \angle in {0, 30, ..., 330} 
    \draw[line width=1pt] (\angle:0.82cm) -- (\angle:1cm);
  \foreach \angle in {0,90,180,270}
    \draw[line width=1.3pt] (\angle:0.75cm) -- (\angle:1cm);
  \draw[line width=1.6pt] (0,0) -- (90-30*#4:0.6cm); % the hand 
  \end{scope}
}
% Default on latex.ltx: \fboxsep = 3pt \fboxrule = .4pt
\fboxsep = 3pt \fboxrule = 1.5pt 
\newcounter{itimesj}
\fbox{%
\begin{tikzpicture}[scale=0.5]
\begin{scope}[line cap=round]
\foreach \i in {0,...,11}
  \foreach \j in {0,...,11} {
    % We use round by caution, because we are getting real numbers
    \pgfmathparse{round(mod(\i*\j,12))} 
    % \pgfmathresult = 3.0 gives itimesj = 3
    \pgfmathsetcounter{itimesj}{\pgfmathresult} 
    \clock{\i}{-\j}{clock\theitimesj}{\i*\j}; 
  }
\end{scope}

    \draw (-2.7,2.6) node [black]{$\Delta_1$};
  \end{scope}
  
%%%%%%%%%%%%%%%%%%%%%%%%%%%%%%  FILTRE 2


  \begin{scope}[xshift=1.9 cm,yshift=13.5cm, scale=0.5]
    

%%%%%%%%%%%%%%%%%%%%%%%%%%%%%%    BOUTONS ROTATIF
%      Carrés gris
  \fill[gray!50!] (0,0) rectangle (3,3);
  \fill[gray] (1,0) rectangle (3,2);
  \fill[gray!50!black] (2,0) rectangle (3,1);
%%%%%%%%%


\newcommand{\clock}[4]{%
  \begin{scope}[xshift=2.25*#1cm,yshift=2.25*#2cm]
  % \filldraw [fill=#3, line width=1.6pt] (0,0) circle (1cm); % simple fill (unused)
  % \draw[fill] (0,0) circle (1mm); % border (unused)
  \shadedraw [inner color=#3!30!white, outer color=#3!90!black, 
    line width=1.6pt] (0,0) circle (1cm); % disk with shadow and border
  \foreach \angle in {0, 30, ..., 330} 
    \draw[line width=1pt] (\angle:0.82cm) -- (\angle:1cm);
  \foreach \angle in {0,90,180,270}
    \draw[line width=1.3pt] (\angle:0.75cm) -- (\angle:1cm);
  \draw[line width=1.6pt] (0,0) -- (90-30*#4:0.6cm); % the hand 
  \end{scope}
}
% Default on latex.ltx: \fboxsep = 3pt \fboxrule = .4pt
\fboxsep = 3pt \fboxrule = 1.5pt 
\newcounter{itimesj}
\fbox{%
\begin{tikzpicture}[scale=0.5]
\begin{scope}[line cap=round]
\foreach \i in {0,...,11}
  \foreach \j in {0,...,11} {
    % We use round by caution, because we are getting real numbers
    \pgfmathparse{round(mod(\i*\j,12))} 
    % \pgfmathresult = 3.0 gives itimesj = 3
    \pgfmathsetcounter{itimesj}{\pgfmathresult} 
    \clock{\i}{-\j}{clock\theitimesj}{\i*\j}; 
  }
\end{scope}

    \draw (-2.7,2.6) node [black]{f$_2$};
  \end{scope}
  \begin{scope}[xshift=1.9 cm,yshift=11.7cm, scale=0.5]
    

%%%%%%%%%%%%%%%%%%%%%%%%%%%%%%    BOUTONS ROTATIF
%      Carrés gris
  \fill[gray!50!] (0,0) rectangle (3,3);
  \fill[gray] (1,0) rectangle (3,2);
  \fill[gray!50!black] (2,0) rectangle (3,1);
%%%%%%%%%


\newcommand{\clock}[4]{%
  \begin{scope}[xshift=2.25*#1cm,yshift=2.25*#2cm]
  % \filldraw [fill=#3, line width=1.6pt] (0,0) circle (1cm); % simple fill (unused)
  % \draw[fill] (0,0) circle (1mm); % border (unused)
  \shadedraw [inner color=#3!30!white, outer color=#3!90!black, 
    line width=1.6pt] (0,0) circle (1cm); % disk with shadow and border
  \foreach \angle in {0, 30, ..., 330} 
    \draw[line width=1pt] (\angle:0.82cm) -- (\angle:1cm);
  \foreach \angle in {0,90,180,270}
    \draw[line width=1.3pt] (\angle:0.75cm) -- (\angle:1cm);
  \draw[line width=1.6pt] (0,0) -- (90-30*#4:0.6cm); % the hand 
  \end{scope}
}
% Default on latex.ltx: \fboxsep = 3pt \fboxrule = .4pt
\fboxsep = 3pt \fboxrule = 1.5pt 
\newcounter{itimesj}
\fbox{%
\begin{tikzpicture}[scale=0.5]
\begin{scope}[line cap=round]
\foreach \i in {0,...,11}
  \foreach \j in {0,...,11} {
    % We use round by caution, because we are getting real numbers
    \pgfmathparse{round(mod(\i*\j,12))} 
    % \pgfmathresult = 3.0 gives itimesj = 3
    \pgfmathsetcounter{itimesj}{\pgfmathresult} 
    \clock{\i}{-\j}{clock\theitimesj}{\i*\j}; 
  }
\end{scope}

    \draw (-2.7,2.6) node [black]{$\Delta_2$};
  \end{scope}

%%%%%%%%%%%%%%%%%%%%%%%%%%%%%%  FILTRE 3

  \begin{scope}[xshift=1.9 cm,yshift=9cm, scale=0.5]
    

%%%%%%%%%%%%%%%%%%%%%%%%%%%%%%    BOUTONS ROTATIF
%      Carrés gris
  \fill[gray!50!] (0,0) rectangle (3,3);
  \fill[gray] (1,0) rectangle (3,2);
  \fill[gray!50!black] (2,0) rectangle (3,1);
%%%%%%%%%


\newcommand{\clock}[4]{%
  \begin{scope}[xshift=2.25*#1cm,yshift=2.25*#2cm]
  % \filldraw [fill=#3, line width=1.6pt] (0,0) circle (1cm); % simple fill (unused)
  % \draw[fill] (0,0) circle (1mm); % border (unused)
  \shadedraw [inner color=#3!30!white, outer color=#3!90!black, 
    line width=1.6pt] (0,0) circle (1cm); % disk with shadow and border
  \foreach \angle in {0, 30, ..., 330} 
    \draw[line width=1pt] (\angle:0.82cm) -- (\angle:1cm);
  \foreach \angle in {0,90,180,270}
    \draw[line width=1.3pt] (\angle:0.75cm) -- (\angle:1cm);
  \draw[line width=1.6pt] (0,0) -- (90-30*#4:0.6cm); % the hand 
  \end{scope}
}
% Default on latex.ltx: \fboxsep = 3pt \fboxrule = .4pt
\fboxsep = 3pt \fboxrule = 1.5pt 
\newcounter{itimesj}
\fbox{%
\begin{tikzpicture}[scale=0.5]
\begin{scope}[line cap=round]
\foreach \i in {0,...,11}
  \foreach \j in {0,...,11} {
    % We use round by caution, because we are getting real numbers
    \pgfmathparse{round(mod(\i*\j,12))} 
    % \pgfmathresult = 3.0 gives itimesj = 3
    \pgfmathsetcounter{itimesj}{\pgfmathresult} 
    \clock{\i}{-\j}{clock\theitimesj}{\i*\j}; 
  }
\end{scope}

    \draw (-2.7,2.6) node [black]{f$_3$};
  \end{scope}
  \begin{scope}[xshift=1.9 cm,yshift=7.2cm, scale=0.5]
    

%%%%%%%%%%%%%%%%%%%%%%%%%%%%%%    BOUTONS ROTATIF
%      Carrés gris
  \fill[gray!50!] (0,0) rectangle (3,3);
  \fill[gray] (1,0) rectangle (3,2);
  \fill[gray!50!black] (2,0) rectangle (3,1);
%%%%%%%%%


\newcommand{\clock}[4]{%
  \begin{scope}[xshift=2.25*#1cm,yshift=2.25*#2cm]
  % \filldraw [fill=#3, line width=1.6pt] (0,0) circle (1cm); % simple fill (unused)
  % \draw[fill] (0,0) circle (1mm); % border (unused)
  \shadedraw [inner color=#3!30!white, outer color=#3!90!black, 
    line width=1.6pt] (0,0) circle (1cm); % disk with shadow and border
  \foreach \angle in {0, 30, ..., 330} 
    \draw[line width=1pt] (\angle:0.82cm) -- (\angle:1cm);
  \foreach \angle in {0,90,180,270}
    \draw[line width=1.3pt] (\angle:0.75cm) -- (\angle:1cm);
  \draw[line width=1.6pt] (0,0) -- (90-30*#4:0.6cm); % the hand 
  \end{scope}
}
% Default on latex.ltx: \fboxsep = 3pt \fboxrule = .4pt
\fboxsep = 3pt \fboxrule = 1.5pt 
\newcounter{itimesj}
\fbox{%
\begin{tikzpicture}[scale=0.5]
\begin{scope}[line cap=round]
\foreach \i in {0,...,11}
  \foreach \j in {0,...,11} {
    % We use round by caution, because we are getting real numbers
    \pgfmathparse{round(mod(\i*\j,12))} 
    % \pgfmathresult = 3.0 gives itimesj = 3
    \pgfmathsetcounter{itimesj}{\pgfmathresult} 
    \clock{\i}{-\j}{clock\theitimesj}{\i*\j}; 
  }
\end{scope}

    \draw (-2.7,2.6) node [black]{$\Delta_3$};
  \end{scope}
  \begin{scope}[xshift=1.9 cm,yshift=5.4cm, scale=0.5]
    

%%%%%%%%%%%%%%%%%%%%%%%%%%%%%%    BOUTONS ROTATIF
%      Carrés gris
  \fill[gray!50!] (0,0) rectangle (3,3);
  \fill[gray] (1,0) rectangle (3,2);
  \fill[gray!50!black] (2,0) rectangle (3,1);
%%%%%%%%%


\newcommand{\clock}[4]{%
  \begin{scope}[xshift=2.25*#1cm,yshift=2.25*#2cm]
  % \filldraw [fill=#3, line width=1.6pt] (0,0) circle (1cm); % simple fill (unused)
  % \draw[fill] (0,0) circle (1mm); % border (unused)
  \shadedraw [inner color=#3!30!white, outer color=#3!90!black, 
    line width=1.6pt] (0,0) circle (1cm); % disk with shadow and border
  \foreach \angle in {0, 30, ..., 330} 
    \draw[line width=1pt] (\angle:0.82cm) -- (\angle:1cm);
  \foreach \angle in {0,90,180,270}
    \draw[line width=1.3pt] (\angle:0.75cm) -- (\angle:1cm);
  \draw[line width=1.6pt] (0,0) -- (90-30*#4:0.6cm); % the hand 
  \end{scope}
}
% Default on latex.ltx: \fboxsep = 3pt \fboxrule = .4pt
\fboxsep = 3pt \fboxrule = 1.5pt 
\newcounter{itimesj}
\fbox{%
\begin{tikzpicture}[scale=0.5]
\begin{scope}[line cap=round]
\foreach \i in {0,...,11}
  \foreach \j in {0,...,11} {
    % We use round by caution, because we are getting real numbers
    \pgfmathparse{round(mod(\i*\j,12))} 
    % \pgfmathresult = 3.0 gives itimesj = 3
    \pgfmathsetcounter{itimesj}{\pgfmathresult} 
    \clock{\i}{-\j}{clock\theitimesj}{\i*\j}; 
  }
\end{scope}

    \draw (-2.7,2.6) node [black]{$\Delta f$};
  \end{scope}

%%%%%%%%%%%%%%%%%%%%%%%%%%%%%%  AMPLIFICATION
  \begin{scope}[xshift=1.9 cm,yshift=2.8cm, scale=0.5]
    

%%%%%%%%%%%%%%%%%%%%%%%%%%%%%%    BOUTONS ROTATIF
%      Carrés gris
  \fill[gray!50!] (0,0) rectangle (3,3);
  \fill[gray] (1,0) rectangle (3,2);
  \fill[gray!50!black] (2,0) rectangle (3,1);
%%%%%%%%%


\newcommand{\clock}[4]{%
  \begin{scope}[xshift=2.25*#1cm,yshift=2.25*#2cm]
  % \filldraw [fill=#3, line width=1.6pt] (0,0) circle (1cm); % simple fill (unused)
  % \draw[fill] (0,0) circle (1mm); % border (unused)
  \shadedraw [inner color=#3!30!white, outer color=#3!90!black, 
    line width=1.6pt] (0,0) circle (1cm); % disk with shadow and border
  \foreach \angle in {0, 30, ..., 330} 
    \draw[line width=1pt] (\angle:0.82cm) -- (\angle:1cm);
  \foreach \angle in {0,90,180,270}
    \draw[line width=1.3pt] (\angle:0.75cm) -- (\angle:1cm);
  \draw[line width=1.6pt] (0,0) -- (90-30*#4:0.6cm); % the hand 
  \end{scope}
}
% Default on latex.ltx: \fboxsep = 3pt \fboxrule = .4pt
\fboxsep = 3pt \fboxrule = 1.5pt 
\newcounter{itimesj}
\fbox{%
\begin{tikzpicture}[scale=0.5]
\begin{scope}[line cap=round]
\foreach \i in {0,...,11}
  \foreach \j in {0,...,11} {
    % We use round by caution, because we are getting real numbers
    \pgfmathparse{round(mod(\i*\j,12))} 
    % \pgfmathresult = 3.0 gives itimesj = 3
    \pgfmathsetcounter{itimesj}{\pgfmathresult} 
    \clock{\i}{-\j}{clock\theitimesj}{\i*\j}; 
  }
\end{scope}

    \draw (-2.7,2.6) node [black]{G};
  \end{scope}

\end{scope}


%
%

\def\separ{1.1}
\def\cote{1}
%%%%%%%%%%%%%%%%%%%%%%%%%%%%%%                          MENU  DES GRAPHES

                   %  Panneau
  \begin{scope}[xshift=0 cm,yshift=0cm]
    \fill[gray,draw=gray!10!] (-0.1, -0.2) rectangle (9.2,0.9);
    \fill[panneauControles]
      (-0.1, -0.1) rectangle (9.1, 0.8);
  \end{scope}

%%%%%%%%%%%%%%%%%%%%%%%%%%%%%%      POINT DE VUE

  \begin{scope}[xshift=2 cm,yshift=0cm, scale=0.7]
      \begin{scope}[xshift=0.0 cm] %              Implicite
        \fill[boutonEteint] (0.0,0.0) rectangle (\cote,\cote);
        \draw[styleEteint] (0.1,0.7)--(0.9,0.3);
        \draw[styleEteint] (0.5,0.7)--(0.5,0.3);
        \draw[styleEteint] (0.3,0.5)--(0.7,0.5);
      \end{scope}
      \begin{scope}[xshift=\separ cm] %              Imaginaire
        \fill[boutonEteint] (0.0,0.0) rectangle (\cote,\cote);
        \draw (0.5,0.7) node {$\mc{I}$\it{m}};
        \draw[styleEteint] (0.1,0.3)--(0.9,0.3);
        \draw[styleEteint] (0.5,0.1)--(0.5,0.5);
      \end{scope}
      \begin{scope}[xshift=2*\separ cm] %              Réel
        \fill[boutonEteint] (0.0,0.0) rectangle (\cote,\cote);
        \draw (0.5,0.7) node {$\mc{R}$\it{e}};
        \draw[styleEteint] (0.1,0.3)--(0.9,0.3);
        \draw[styleEteint] (0.5,0.1)--(0.5,0.5);
      \end{scope}
  \end{scope}

%%%%%%%%%%%%%%%%%%%%%%%%%%%%%%      TRACÉ DE LA COURBE

  \begin{scope}[xshift=5 cm,yshift=0cm, scale=0.7]
     % \begin{scope}[xshift=0.0 cm] %              points
          %  dotted, loosely dotted, densely dotted
     %   \fill[boutonEteint] (0.0,0.0) rectangle (\cote,\cote);
     %   \draw[styleEteint, loosely dotted] (0.1,0.5) sin (0.3,0.9) cos (0.5,0.5) sin (0.7,0.1) cos (0.9,0.5);
     % \end{scope}
      \begin{scope}[xshift=0 cm] %              relié
        \fill[boutonEteint] (0.0,0.0) rectangle (\cote,\cote);
        \draw[styleEteint] (0.1,0.5) sin (0.3,0.9) cos (0.5,0.5) sin (0.7,0.1) cos (0.9,0.5);
      \end{scope}
      \begin{scope}[xshift=1.5 cm] %              vecteur
        \fill[boutonEteint] (0.0,0.0) rectangle (\cote,\cote);
        \draw[styleEteint] (0.3,0.6) -- (0.2,0.7);
        \draw[styleEteint] (0.5,0.5) -- (0.25,0.45);
        \draw[styleEteint] (0.7,0.4) -- (0.4,0.25);
        \draw[styleEteint] (0.9,0.3) -- (0.7,0.1);
      \end{scope}
  \begin{scope}[xshift=3 cm] %	tracé des axes
        \fill[boutonEteint] (0.0,0.0) rectangle (\cote,\cote);
        \draw[->,styleEteint] (0.4,0.2)--(0.4,0.9);
        \draw[->,styleEteint] (0.1,0.5)--(0.9,0.5);
      \end{scope}
    \end{scope}
%%%%%%%%%%%%%%%%%%%%%%%%%%%%%%%%%%%%%%%%%%%%%%%%

%
%

%%%%%%%%%%%%%%%%%%%%%%%%%%%%%%    BOUTONS ROTATIF
%      Carrés gris
\begin{comment}
  \begin{scope}[xshift=0 cm,yshift=15cm, scale=0.5]
  \fill[gray!50!] (0,0) rectangle (3,3);
  \fill[gray] (1,0) rectangle (3,2);
  \fill[gray!50!black] (2,0) rectangle (3,1);
  \end{scope}
\end{comment}

%%%%%%%%%%%%%%%%%%%%%%%%%%%%%     BOUTONS SELECTIF
%      Grille
\begin{scope}[xshift=0 cm,yshift=4cm]
    \begin{scope}[xshift=3.2 cm,yshift=0.1cm] %      Grille
      \fill[boutonSelect] (-0.075,0) rectangle (1.075,4.6);
      \foreach \i in {0,1.15,...,5} {\draw[boutonSelect] (-0.075,\i)--(1.075,\i);}
      \draw[boutonSelect] (-0.075,0)--(-0.075,4.6);
      \draw[boutonSelect] (1.075,0)--(1.075,4.6);
      \begin{scope}[yshift=3.54 cm] %              Constant
        \draw[boutonSelect] (0.1,0.7)--(0.9,0.7);
      \end{scope}
      \begin{scope}[yshift=2.37 cm] %              Carré NS
        \draw[boutonSelect] (0.1,0.5)--(0.1,0.9)--(0.6,0.9)--(0.6,0.1)--(0.9,0.1)--(0.9,0.5);
      \end{scope}
      \begin{scope}[yshift=1.24 cm] %           Triangle NS
        \draw[boutonSelect] (0.1,0.5)--(0.4,0.9)--(0.6,0.1)--(0.9,0.5);
      \end{scope}
      \begin{scope}[yshift=0.1 cm] %               Sinus
        \draw[boutonSelect] (0.1,0.5) sin (0.3,0.9) cos (0.5,0.5) sin (0.7,0.1) cos (0.9,0.5);
      \end{scope}
    \end{scope}
\end{scope}

\begin{scope}[xshift=0 cm,yshift=4cm]
  \begin{scope}[xshift=0.2 cm,yshift=8cm]
    \begin{scope}[xshift=3.2 cm,yshift=0cm]
      \fill[boutonSelect] (-0.075,0) rectangle (1.075,5.8);
      \foreach \i in {0,1.15,...,5} {\draw[boutonSelect] (-0.075,\i)--(1.075,\i);}
      \draw[boutonSelect] (-0.075,0)--(-0.075,4.6);\draw[boutonSelect] (1.075,0)--(1.075,4.6);
      \begin{scope}[yshift=4.7 cm] % 0
        \draw[boutonSelect] (0.1,0.3)--(0.9,0.3);
        \draw[boutonSelect] (0.5,0.3)--(0.5,0.7);
      \end{scope}
      \begin{scope}[yshift=3.5 cm] % 1
        \draw[boutonSelect] (0.2,0.3)--(0.2,0.7);\draw[boutonSelect] (0.5,0.3)--(0.5,0.7);
        \draw[boutonSelect] (0.8,0.3)--(0.8,0.7); 
        \draw[boutonSelect] (0.1,0.3)--(0.9,0.3);
      \end{scope}
      \begin{scope}[xshift=0.1 cm,yshift=2.45 cm] % Gaussienne
          \draw [boutonSelect, domain=-0.4:0.4, samples=80]
            plot (\x+0.4, {0.1+0.6*exp(-100 *\x * \x)}) ;
      \end{scope}
      \begin{scope}[xshift=0.1 cm,yshift=1.33 cm] % Lorentzienne
          \draw [boutonSelect, domain=-0.4:0.4, samples=80]
            plot (\x+0.4, {0.7/(1 + 100 * \x * \x)}) ;
      \end{scope}
      \begin{scope}[xshift=0.1 cm,yshift=0.3 cm] % sinc
          \draw [boutonSelect, domain=-3.55:3.55, samples=80]
            plot (0.1*\x+0.4, {(1 - \x * \x * ( 0.31) + \x * \x * \x * \x * (0.05)
             - \x * \x * \x * \x * \x * \x * ( 0.0023)-0.4}) ;
      \end{scope}
    \end{scope}
  \end{scope}

\end{scope}
\begin{scope}[xshift=-2 cm,yshift=2cm]
    \begin{scope}[xshift=3.2 cm]  %      Grille
      \fill[boutonSelect] (-0.075,0) rectangle (1.075,4.6);
      \foreach \i in {0,1.15,...,5} {\draw[boutonSelect] (-0.075,\i)--(1.075,\i);}
      \draw[boutonSelect] (-0.075,0)--(-0.075,4.6);\draw[boutonSelect] (1.075,0)--(1.075,4.6);
      \begin{scope}[yshift=3.8 cm] % Constant
          \draw[boutonSelect] (0.1,0.5)--(0.9,0.5);
      \end{scope}
      \begin{scope}[yshift=2.4 cm] % dirac
          \draw[boutonSelect, >=latex, ->] (0.3,0.15)--(0.3,0.85);
          \draw[boutonSelect, >=latex, ->] (0.7,0.15)--(0.7,0.85);
          \draw[boutonSelect] (0.1,0.15)--(0.9,0.15);
      \end{scope}
      \begin{scope}[yshift=1.25 cm] % Sinus
          \draw[boutonSelect] (0.1,0.5) sin (0.3,0.9) cos (0.5,0.5) sin (0.7,0.1) cos (0.9,0.5);
      \end{scope}
  \begin{scope}[xshift=0.3 cm,yshift=0.15cm] % spirale
  \draw [boutonSelect, domain=0.08:0.9, samples=80]
  plot (\x, {0.6+0.35*sin(15*\x r)}, {0.6+0.35*cos(15*\x r)}) ;
  \end{scope}
    \end{scope}
\end{scope}
% 0
% 1
\begin{comment}
  \begin{scope}[xshift=0 cm,yshift=1.15cm] % Constant
  \draw[thick] (0.1,0.7)--(0.9,0.7);
  \end{scope}
  \begin{scope}[xshift=1 cm,yshift=1cm] % Carré
  \draw[thick] (0.1,0.5)--(0.1,0.9)--(0.5,0.9)--(0.5,0.1)--(0.9,0.1)--(0.9,0.1)--(0.9,0.5);
  \end{scope}
  \begin{scope}[xshift=2 cm,yshift=1cm] % Triangle
  \draw[thick] (0.1,0.5)--(0.3,0.9)--(0.7,0.1)--(0.9,0.5);
  \end{scope}
  \begin{scope}[xshift=3 cm,yshift=1cm] % Sinus
  \draw[thick] (0.1,0.5) sin (0.3,0.9) cos (0.5,0.5) sin (0.7,0.1) cos (0.9,0.5);
  \end{scope}
% 2
  \begin{scope}[xshift=0 cm,yshift=0cm] % Constant
  \draw[very thick] (0.1,0.7)--(0.9,0.7);
  \end{scope}
  \begin{scope}[xshift=1.2 cm,yshift=0.1cm] % Carré NS  --(0.2,0.5)  --(0.8,0.5)
  \draw[very thick] (0.1,0.5)--(0.1,0.9)--(0.6,0.9)--(0.6,0.1)--(0.9,0.1)--(0.9,0.5);
  \end{scope}
  \begin{scope}[xshift=2.3 cm,yshift=0cm] % Triangle NS
  \draw[very thick] (0.1,0.5)--(0.4,0.9)--(0.6,0.1)--(0.9,0.5);
  \end{scope}
  \begin{scope}[xshift=3.45 cm,yshift=0cm] % Sinus
  \draw[very thick] (0.1,0.5) sin (0.3,0.9) cos (0.5,0.5) sin (0.7,0.1) cos (0.9,0.5);
  \end{scope}
% 3
  \begin{scope}[xshift=0 cm,yshift=3.45cm] % Constant
  \draw[thick] (0.1,0.7)--(0.9,0.7);
  \end{scope}
  \begin{scope}[xshift=1 cm,yshift=3cm] % Carré NP
  \draw[thick] (0.1,0.3)--(0.3,0.3)--(0.3,0.7)--(0.7,0.7)--(0.7,0.3)--(0.9,0.3);
  \end{scope}
  \begin{scope}[xshift=2 cm,yshift=3cm] % Triangle NP
  \draw[thick] (0.1,0.3)--(0.3,0.3)--(0.5,0.7)--(0.7,0.3)--(0.9,0.3);
  \end{scope}
  \begin{scope}[xshift=3 cm,yshift=3cm] % Sinus NP
  %\draw[thick] (0.1,0.5) cos (0.9,0.5);
  \draw[thick] (0.2,0.3) cos (0.35,0.5) sin (0.5,0.7) cos (0.65,0.5) sin (0.8,0.3) ;
  \draw[thick] (0.1,0.3)--(0.2,0.3); \draw[thick] (0.8,0.3)--(0.9,0.3);
  \end{scope}
% 4
  \begin{scope}[xshift=0 cm,yshift=1.15cm] % Constant
  \draw[very thick] (0.1,0.5)--(0.9,0.5);
  \end{scope}
  \begin{scope}[xshift=1.2 cm,yshift=1.2cm] % dirac
  \draw[very thick, >=latex, ->] (0.3,0.15)--(0.3,0.85);
  \draw[very thick, >=latex, ->] (0.7,0.15)--(0.7,0.85);
  \draw[very thick] (0.1,0.15)--(0.9,0.15);
  \end{scope}
  \begin{scope}[xshift=3.7 cm,yshift=1.2cm] % spirale
  \draw [very thick, domain=0.08:0.9, samples=80]
  plot (\x, {0.6+0.35*sin(15*\x r)}, {0.6+0.35*cos(15*\x r)}) ;
  \end{scope}
  \begin{scope}[xshift=2.3 cm,yshift=1.2cm] % Sinus
  \draw[very thick] (0.1,0.5) sin (0.3,0.9) cos (0.5,0.5) sin (0.7,0.1) cos (0.9,0.5);
  \end{scope}
\end{comment}



%
%

%%%%%%%%%%%%%%%%%%%%%%%%%%%%%%    BOUTONS ROTATIF
%      Carrés gris
  \fill[gray!50!] (0,0) rectangle (3,3);
  \fill[gray] (1,0) rectangle (3,2);
  \fill[gray!50!black] (2,0) rectangle (3,1);
%%%%%%%%%


\newcommand{\clock}[4]{%
  \begin{scope}[xshift=2.25*#1cm,yshift=2.25*#2cm]
  % \filldraw [fill=#3, line width=1.6pt] (0,0) circle (1cm); % simple fill (unused)
  % \draw[fill] (0,0) circle (1mm); % border (unused)
  \shadedraw [inner color=#3!30!white, outer color=#3!90!black, 
    line width=1.6pt] (0,0) circle (1cm); % disk with shadow and border
  \foreach \angle in {0, 30, ..., 330} 
    \draw[line width=1pt] (\angle:0.82cm) -- (\angle:1cm);
  \foreach \angle in {0,90,180,270}
    \draw[line width=1.3pt] (\angle:0.75cm) -- (\angle:1cm);
  \draw[line width=1.6pt] (0,0) -- (90-30*#4:0.6cm); % the hand 
  \end{scope}
}
% Default on latex.ltx: \fboxsep = 3pt \fboxrule = .4pt
\fboxsep = 3pt \fboxrule = 1.5pt 
\newcounter{itimesj}
\fbox{%
\begin{tikzpicture}[scale=0.5]
\begin{scope}[line cap=round]
\foreach \i in {0,...,11}
  \foreach \j in {0,...,11} {
    % We use round by caution, because we are getting real numbers
    \pgfmathparse{round(mod(\i*\j,12))} 
    % \pgfmathresult = 3.0 gives itimesj = 3
    \pgfmathsetcounter{itimesj}{\pgfmathresult} 
    \clock{\i}{-\j}{clock\theitimesj}{\i*\j}; 
  }
\end{scope}

%


\def\separ{1.1}
\def\cote{1}
%%%%%%%%%%%%%%%%%%%%%%%%%%%%%%             MENU  FILTRAGE  SELECTIFS

                   %  Panneau
  \begin{scope}[xshift=0 cm,yshift=0cm]
    \fill[gray,draw=gray!10!] (-5.7, -0.2) rectangle (8.8,0.9);
    \fill[panneauControles]
      (-5.6, -0.1) rectangle (8.8, 0.8);
  \end{scope}

%%%%%%%%%%%%%%%%%%%%%%%%%%%%%%      PASSE  BAS

  \begin{scope}[xshift=-5 cm,yshift=0cm, scale=0.7]
   % \begin{scope}[xshift=2 cm,yshift=0cm] %      Grille
     % \fill[boutonEteint] (0.1,0.0) rectangle (4.6,1);
     % \foreach \i in {1.1, 2.3,...,4} {\draw[boutonEteint] (\i,-0.0)--(\i,1.0);}
      \begin{scope}[xshift=0 cm] %             éteint
        %\fill[styleEteint] (0.5,0.5) circle (\cote / 2);
        \draw[styleEteint] (0.5,0.5) node [black]{\bf{1}};
      \end{scope}
      \begin{scope}[xshift=\separ cm] %              Symétrique
        \shadedraw[bouton] (0.0,0.0) rectangle (\cote,\cote);
        \draw[styleEteint] (0.1,0.3)--(0.3,0.3)--(0.3,0.7)--(0.7,0.7)--(0.7,0.3)--(0.9,0.3);
      \end{scope}
      \begin{scope}[xshift=2*\separ cm] %              Dissymétrie 1
        \shadedraw[bouton] (0.0,0.0) rectangle (1,1);
        \draw[styleEteint] (0.1,0.3)--(0.3,0.3)--(0.3,0.7)--(0.9,0.7);
      \end{scope}
      \begin{scope}[xshift=3*\separ cm] %              Dissymétrie 2
        \shadedraw[bouton] (0.0,0.0) rectangle (1,1);
        \draw[styleEteint] (0.1,0.7)--(0.7,0.7)--(0.7,0.3)--(0.9,0.3);
      \end{scope}
      \begin{scope}[xshift=4*\separ cm] %              Inverse
        \shadedraw[bouton] (0.0,0.0) rectangle (1,1);
        \draw (0.5,0.5) node [black]{inv};
      \end{scope}
   % \end{scope}
  \end{scope}


%%%%%%%%%%%%%%%%%%%%%%%%%%%%%%      PASSE  HAUT

  \begin{scope}[xshift=-0.5 cm,yshift=0cm, scale=0.7]
      \begin{scope}[xshift=0 cm] %             éteint
        %\fill[boutonEteint] (0.5,0.5) circle (\cote / 2);
        \draw (0.5,0.5) node [black]{\bf{2}};
      \end{scope}
      \begin{scope}[xshift=\separ cm] %              Symétrique
        \shadedraw[bouton] (0.0,0.0) rectangle (\cote,\cote);
        \draw[styleEteint] (0.1,0.7)--(0.3,0.7)--(0.3,0.3)--(0.7,0.3)--(0.7,0.7)--(0.9,0.7);
      \end{scope}
      \begin{scope}[xshift=2*\separ cm] %              Dissymétrie 1
        \shadedraw[bouton] (0.0,0.0) rectangle (\cote,\cote);
        \draw[styleEteint] (0.1,0.7)--(0.3,0.7)--(0.3,0.3)--((0.5,0.3);
        \draw[styleEteint] (0.5,0.7)--((0.9,0.7);
      \end{scope}
      \begin{scope}[xshift=3*\separ cm] %              Dissymétrie 2
        \shadedraw[bouton] (0.0,0.0) rectangle (\cote,\cote);
        \draw[styleEteint] (0.1,0.7)--(0.5,0.7);
        \draw[styleEteint] (0.5,0.3)--(0.7,0.3)--(0.7,0.7)--(0.9,0.7);
      \end{scope}
      \begin{scope}[xshift=4*\separ cm] %              Inverse
        \shadedraw[bouton] (0.0,0.0) rectangle (1,1);
        \draw (0.5,0.5) node [black]{inv};
      \end{scope}
  \end{scope}


%%%%%%%%%%%%%%%%%%%%%%%%%%%%%%      PASSE  BANDE

  \begin{scope}[xshift=4 cm,yshift=0cm, scale=0.7]
      \begin{scope}[xshift=0 cm] %             éteint
       % \fill[boutonEteint] (0.5,0.5) circle (\cote / 2);
        \draw (0.5,0.5) node [black]{\bf{3}};
      \end{scope}
      \begin{scope}[xshift=\separ cm] %              Symétrique
        \shadedraw[bouton] (0.0,0.0) rectangle (\cote,\cote);
        \draw[styleEteint] (0.1,0.3)--(0.2,0.3)--(0.2,0.7)--(0.3,0.7)--(0.3,0.3)--(0.7,0.3)
        --(0.7,0.7)--(0.8,0.7)--(0.8,0.3)--(0.9,0.3);
      \end{scope}
      \begin{scope}[xshift=2*\separ cm] %              Dissymétrie 1
        \shadedraw[bouton] (0.0,0.0) rectangle (\cote,\cote);
        \draw[styleEteint] (0.1,0.3)--(0.2,0.3)--(0.2,0.7)--(0.3,0.7)--(0.3,0.3)--(0.5,0.3);
        \draw[styleEteint] (0.5,0.7)--(0.9,0.7);
      \end{scope}
      \begin{scope}[xshift=3*\separ cm] %              Dissymétrie 2
        \shadedraw[bouton] (0.0,0.0) rectangle (\cote,\cote);
        \draw[styleEteint] (0.1,0.7)--(0.5,0.7);
        \draw[styleEteint] (0.5,0.3)--(0.7,0.3)--(0.7,0.7)--(0.8,0.7)--(0.8,0.3)--(0.9,0.3);
      \end{scope}
      \begin{scope}[xshift=4*\separ cm] %              Inverse
        \shadedraw[bouton] (0.0,0.0) rectangle (\cote,\cote);
        \draw (0.5,0.5) node [black]{inv};
      \end{scope}
  \end{scope}

%%%%%%%%%%%%%%%%%%%%%%%%%%%%%%%%%%%%%%%%%%%%%

%%%%%%%%%%%%%%%%%%%%%%%%%%%%%%                          MENU  DES GRAPHES

                   %  Panneau
  \begin{scope}[xshift=8 cm,yshift=0cm]
    \fill[gray,draw=gray!10!] (1, -0.2) rectangle (9.2,0.9);
    \fill[panneauControles]
      (1, -0.1) rectangle (9.1, 0.8);
  \end{scope}

%%%%%%%%%%%%%%%%%%%%%%%%%%%%%%      POINT DE VUE
%\shadedraw[bouton]

  \begin{scope}[xshift=10 cm,yshift=0cm, scale=0.7]
      \begin{scope}[xshift=0.0 cm] %              ImpliciteGoldenrod
        \shadedraw[bouton] (0.0,0.0) rectangle (\cote,\cote);
        \draw[styleEteint] (0.1,0.7)--(0.9,0.3);
        \draw[styleEteint] (0.5,0.7)--(0.5,0.3);
        \draw[styleEteint] (0.3,0.5)--(0.7,0.5);
      \end{scope}
      \begin{scope}[xshift=\separ cm] %              Imaginaire
        \shadedraw[bouton] (0.0,0.0) rectangle (\cote,\cote);
        \draw (0.5,0.7) node {$\mc{I}$\it{m}};
        \draw[styleEteint] (0.1,0.3)--(0.9,0.3);
        \draw[styleEteint] (0.5,0.1)--(0.5,0.5);
      \end{scope}
      \begin{scope}[xshift=2*\separ cm] %              Réel
        \shadedraw[bouton] (0.0,0.0) rectangle (\cote,\cote);
        \draw (0.5,0.7) node {$\mc{R}$\it{e}};
        \draw[styleEteint] (0.1,0.3)--(0.9,0.3);
        \draw[styleEteint] (0.5,0.1)--(0.5,0.5);
      \end{scope}
  \end{scope}

%%%%%%%%%%%%%%%%%%%%%%%%%%%%%%      TRACÉ DE LA COURBE

  \begin{scope}[xshift=13 cm,yshift=0cm, scale=0.7]
      \begin{scope}[xshift=0 cm] %              relié
        \shadedraw[bouton] (0.0,0.0) rectangle (\cote,\cote);
        \draw[styleEteint] (0.1,0.5) sin (0.3,0.9) cos (0.5,0.5) sin (0.7,0.1) cos (0.9,0.5);
      \end{scope}
      \begin{scope}[xshift=1.7 cm] %              vecteur
        \shadedraw[bouton] (0.0,0.0) rectangle (\cote,\cote);
        \draw[styleEteint] (0.1,0.5) -- (0.1,0.6);
        \draw[styleEteint] (0.25,0.5) -- (0.25,0.9);
        \draw[styleEteint] (0.4,0.5) -- (0.4,0.7);
        \draw[styleEteint] (0.55,0.5) -- (0.55,0.4);
        \draw[styleEteint] (0.7,0.5) -- (0.7,0.1);
        \draw[styleEteint] (0.85,0.5) -- (0.85,0.25);
      \end{scope}
      \begin{scope}[xshift=3.4 cm] %	tracé des axes
        \shadedraw[bouton] (0.0,0.0) rectangle (\cote,\cote);
        \draw[->,styleEteint] (0.4,0.2)--(0.4,0.9);
        \draw[->,styleEteint] (0.1,0.5)--(0.9,0.5);
      \end{scope}
    \end{scope}
%%%%%%%%%%%%%%%%%%%%%%%%%%%%%%%%%%%%%%%%%%%%%%%%

%

\end{tikzpicture}
%%%%%%%%%%%%%%%%%%%%%%%%%%%%%%%%%%%%%%%%%%%%%%%%%%%%%%%%%%%%%%%%%%%%%%%%%%%%%%%%%%
\end{document}
%%%%%%%%%%%%%%%%%%%%%%%%%%%%%%%%%%%%%%%%%%%%%%%%%%%%%%%%%%%%%%%%%%%%%%%%%%%%%%%%%%

