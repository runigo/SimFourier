% SimFourier
%
% Simulateur de transformation de fourier
%
% Librement inspiré de
% TeXtronics oscilloscope
% Author: Hugues Vermeiren
% http://www.texample.net/tikz/examples/textronics-oscilloscope/
\documentclass[a4paper,12pt,oneside]{article}
\usepackage{tikz}
%%%<
\usepackage{verbatim}
\usepackage[active,tightpage]{preview}
\usepackage[T1]{fontenc}
\usepackage[utf8]{inputenc}
\usepackage{comment}
%\usepackage[greek, frenchb]{babel}
\PreviewEnvironment{tikzpicture}
\setlength\PreviewBorder{0pt}%
%%%>
\begin{comment}
:Title: TeXtronics oscilloscope
:Tags: Styles;Scopes;Physics; Fun
:Author: Hugues Vermeiren
:Slug: textronics-oscilloscope
The TeXtronics oscilloscope! It is a basic 2-channels oscilloscope,
the kind of machine most students have used in their labs.

Ideally, it should be parameterized so as to be able to easily change
the signals and the positions of the different controls.
\end{comment}

\begin{document}
\def\scl{1}%scaling factor of the picture
\sffamily
\begin{tikzpicture}[scale=\scl]

%      Carrés gris
  \begin{scope}[xshift=0 cm,yshift=0cm]
  \fill[gray!50!] (0,0) rectangle (3,3);
  \fill[gray] (1,0) rectangle (3,2);
  \fill[gray!50!black] (2,0) rectangle (3,1);
  \end{scope}

%      Petits boutons
\begin{scope}[xshift=0 cm,yshift=4cm]
    \foreach \i in {0,1,...,4} {\draw[gray] (\i,0)--(\i,4);\draw[gray] (0,\i)--(4,\i);}
  \begin{scope}[xshift=0 cm,yshift=2cm] % Constant
  \draw[thick] (0.1,0.7)--(0.9,0.7);
  \end{scope}
  \begin{scope}[xshift=1 cm,yshift=2cm] % Carré
  \draw[thick] (0.1,0.5)--(0.1,0.9)--(0.5,0.9)--(0.5,0.1)--(0.9,0.1)--(0.9,0.1)--(0.9,0.5);
  \end{scope}
  \begin{scope}[xshift=2 cm,yshift=2cm] % Triangle
  \draw[thick] (0.1,0.5)--(0.3,0.9)--(0.7,0.1)--(0.9,0.5);
  \end{scope}
  \begin{scope}[xshift=3 cm,yshift=2cm] % Sinus
  \draw (0.1,0.5) sin (0.3,0.9) cos (0.5,0.5) sin (0.7,0.1) cos (0.9,0.5);
  \end{scope}
\end{scope}

\end{tikzpicture}
%
%%%%%%%%%%%%%%%%%%%%%%%%%%%%%%%%%%%%%%%%%%%%%%%%%%%%%%%%%%%%%%%%%%%%%%%%%%%%%%%%%%
\end{document}
%%%%%%%%%%%%%%%%%%%%%%%%%%%%%%%%%%%%%%%%%%%%%%%%%%%%%%%%%%%%%%%%%%%%%%%%%%%%%%%%%%
